%%%%%%%%%%%%%%%%%%%%%%%%%%%%%%%%%%%%%%%%%%%%%%%%%
% Custom Commands
%%%%%%%%%%%%%%%%%%%%%%%%%%%%%%%%%%%%%%%%%%%%%%%%%

% \newcommand{\col}[1]{\underline{#1}}
\newcommand{\col}[1]{\boldsymbol{#1}}
\renewcommand{\vec}[1]{\vv{#1}}
\DeclareMathOperator{\sgn}{sgn}
\renewcommand{\bullet}{\boldsymbol{\cdot}}
\DeclareMathOperator{\re}{\mathbb{R}}
\newcommand{\tensor}[1] {\vv{\vv{#1}}}
\newcommand{\e}[1]{{\mathbb E}\left[ #1 \right]}
\newcommand{\expectedvalue}[2]{\operatornamewithlimits{\mathbb E}\limits_{#1} \left[ #2 \right]}
%\DeclareMathOperator*{\max}{max}
%\DeclareMathOperator*{\min}{min}
\DeclareMathOperator*{\argmax}{arg\,max}
\DeclareMathOperator*{\argmin}{arg\,min}
\DeclareMathOperator*{\E}{\mathbb{E}}

% Define a new column type for centered, fixed width, wrapped text
\newcolumntype{C}[1]{>{\centering\arraybackslash}p{#1}}
\newcolumntype{L}[1]{>{\raggedright\arraybackslash}p{#1}}



% ====================================================================
% 1. ÉLÉMENTS THÉORIQUES (Définitions, Théorèmes, Exemples, Preuves)
% ====================================================================

% --- DÉFINITION (Vert) ---
% Syntaxe : \begin{definition}{Titre}{label} ... \end{definition}
\newtcbtheorem[number within=chapter]{definition}{Définition}{
    enhanced,
    colback=green!5!white,      % Fond vert très pâle
    colframe=green!50!black,    % Cadre vert foncé
    coltitle=white,
    fonttitle=\bfseries,
    attach boxed title to top left={yshift=-2mm, xshift=2mm},
    boxed title style={colback=green!50!black, sharp corners},
    breakable,
    terminator sign={ :},
    separator sign={~}
}{def}


% Property env
% \newtheorem{property}{Propriété}[section]
% Boîte pour les DÉFINITIONS (Style bleu)
\newtcbtheorem[number within=chapter]{property}{Propriété}{
    enhanced,           % Permet des styles avancés
    colback=blue!5!white,   % Couleur de fond (bleu très clair)
    colframe=blue!75!black, % Couleur du cadre (bleu foncé)
    fonttitle=\bfseries,    % Titre en gras
    attach boxed title to top left={yshift*=-\tcboxedtitleheight/2, xshift=5mm}, % Titre "posé" sur le cadre
    boxed title style={colback=blue!75!black}
}{def}

% --- THÉORÈME (Bleu) ---
% Partage le compteur des définitions
% Syntaxe : \begin{theorem}{Titre}{label} ... \end{theorem}
\newtcbtheorem[use counter from=definition]{theorem}{Théorème}{
    enhanced,
    colback=blue!5!white,       % Fond bleu très pâle
    colframe=blue!50!black,     % Cadre bleu foncé
    coltitle=white,
    fonttitle=\bfseries,
    attach boxed title to top left={yshift=-2mm, xshift=2mm},
    boxed title style={colback=blue!50!black, sharp corners},
    breakable,
    terminator sign={ :},
    separator sign={~}
}{theo}

% --- EXEMPLE (Gris Sobre) ---
% Syntaxe : \begin{example}{Titre}{label} ... \end{example}
\newtcbtheorem[number within=chapter]{example}{Exemple}{
    enhanced,
    colback=gray!5!white,       % Fond gris très pâle
    colframe=gray!70!black,     % Cadre gris anthracite
    coltitle=white,
    fonttitle=\bfseries,
    attach boxed title to top left={yshift=-2mm, xshift=2mm},
    boxed title style={colback=gray!70!black, sharp corners},
    breakable,
    terminator sign={ :},
    separator sign={~}
}{ex}

% --- PREUVE (Standard Académique) ---
\usepackage{amsthm} % INDISPENSABLE : Charge les commandes de preuve

% Par défaut, amsthm affiche "Proof" en italique.
% On change simplement le nom en "Démonstration" et on ajoute le Gras/Italique ici.
\renewcommand{\proofname}{\textbf{\textit{Démonstration}}}



% ====================================================================
% 2. ÉLÉMENTS MULTIMÉDIAS (Youtube, Colab)
% ====================================================================

% --- Boîte Mère (Structure interne) ---
\newtcolorbox{mediabox}[2][]{
    enhanced,
    width=0.85\textwidth,       % Largeur standardisée
    center,                     % Centrée dans la page
    colback=white,
    colframe=#2,                % Couleur dynamique
    sidebyside,                 % Mode côte-à-côte (Image | Texte)
    sidebyside align=center,
    lefthand width=1.5cm,       % Espace pour l'icone
    boxsep=5pt,
    rounded corners,
    arc=3mm,
    before skip=10pt,
    after skip=10pt,
    #1
}

% --- Commande VIDÉO (Rouge) ---
% Syntaxe : \video{Titre}{URL}
\newcommand{\video}[2]{%
    \begin{mediabox}[colframe=red!75!black]{red!75!black}
        \centering
        \includegraphics[width=12mm]{fig/youtube.png}
        \tcblower
        \textbf{\textsf{Capsule vidéo}} \\
        \textbf{\textit{#1}} \\
        \footnotesize{\url{#2}}
    \end{mediabox}
}

% --- Commande COLAB (Orange) ---
% Syntaxe : \colab{Titre}{URL}
\newcommand{\colab}[2]{%
    \begin{mediabox}[colframe=orange!80!black]{orange!80!black}
        \centering
        \includegraphics[width=12mm]{fig/colab.png}
        \tcblower
        \textbf{\textsf{Exercice de code}} \\
        \textbf{\textit{#1}} \\
        \footnotesize{\url{#2}}
    \end{mediabox}
}

% ====================================================================
% 3. NOTES ET REMARQUES
% ====================================================================

% --- Boîte NOTE (Teal/Bleu canard) ---
% Syntaxe : \note{Titre}{Contenu}
\newtcolorbox{notebox}[1]{
    enhanced,
    colback=teal!5!white,
    colframe=teal!60!black,
    coltitle=white,
    fonttitle=\bfseries,
    title={#1},
    attach boxed title to top left={yshift=-2mm, xshift=2mm},
    boxed title style={colback=teal!60!black, sharp corners},
    breakable,
    before skip=10pt,
    after skip=10pt
}

\newcommand{\note}[2]{%
    \begin{notebox}{#1}
        #2
    \end{notebox}
}



% --- RÉSUMÉ (Violet/Indigo) ---
% Pour conclure une section ("À retenir")
% Syntaxe : \resume{ Formules clés... }
\newtcolorbox{summarybox}{
    enhanced,
    colback=violet!5!white,     % Fond violet très léger
    colframe=violet!60!black,   % Cadre violet foncé
    title={À retenir},
    fonttitle=\bfseries\large,
    attach boxed title to top center={yshift=-2mm}, % Titre centré
    boxed title style={colback=violet!60!black, rounded corners},
    halign=center,              % Contenu centré (idéal pour les équations bilans)
    before skip=20pt,
    after skip=20pt
}
\newcommand{\resume}[1]{\begin{summarybox}#1\end{summarybox}}

%%%%%%%%%%%%%%%%%%%%%%%%%%%%%%%%%%%%%%%%%%%%

% ====================================================================
% VUE D'ENSEMBLE DE CHAPITRE (Introduction C.O.H.P.)
% ====================================================================

% --- Boîte Chapitre (Le contenant) ---
\newtcolorbox{chapterintrobox}{
    enhanced,
    colback=blue!5!gray!5!white,    % Fond gris-bleuté très pâle
    colframe=blue!30!gray!60!black, % Cadre gris-bleu foncé
    title={\Large\textbf{Vue d'ensemble du chapitre}},
    fonttitle=\bfseries,
    attach boxed title to top center={yshift=-3mm},
    boxed title style={
        colback=blue!30!gray!60!black,
        rounded corners,
        frame hidden
    },
    halign=left,
    boxsep=10pt,
    before skip=20pt,
    after skip=30pt,
    breakable
}

% --- Commande interne de style (Ne pas utiliser directement) ---
\newcommand{\introsectionstyle}[2]{%
    \par\vspace{8pt}
    \noindent\textbf{\textsf{\color{blue!40!gray!80!black} \MakeUppercase{#1}}} \par
    \noindent #2 \par
}

% --- Commandes sémantiques (Utilisez celles-ci dans le texte) ---

% 1. CONTEXTE
\newcommand{\introcontext}[1]{%
    \introsectionstyle{Contexte et Motivation}{#1}
}

% 2. OBJECTIFS
\newcommand{\introobjectives}[1]{%
    \introsectionstyle{Objectifs d'apprentissage}{#1}
}

% 3. HYPOTHÈSES
\newcommand{\introhypotheses}[1]{%
    \introsectionstyle{Hypothèses Fondamentales}{#1}
}

% 4. PRÉREQUIS
\newcommand{\introprerequis}[1]{%
    \introsectionstyle{Prérequis et Concepts Clés}{#1}
}

% --- Environnement global ---
\newenvironment{chapterintro}
    {\begin{chapterintrobox}}
    {\end{chapterintrobox}}