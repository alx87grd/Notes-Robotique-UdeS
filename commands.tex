%%%%%%%%%%%%%%%%%%%%%%%%%%%%%%%%%%%%%%%%%%%%%%%%%
% Custom Commands
%%%%%%%%%%%%%%%%%%%%%%%%%%%%%%%%%%%%%%%%%%%%%%%%%

\newcommand{\col}[1]{\underline{#1}}
\DeclareMathOperator{\sgn}{sgn}
\renewcommand{\bullet}{\boldsymbol{\cdot}}
\DeclareMathOperator{\re}{\mathbb{R}}
\newcommand{\tensor}[1] {\vv{\vv{#1}}}
\newcommand{\e}[1]{{\mathbb E}\left[ #1 \right]}
\newcommand{\expectedvalue}[2]{\operatornamewithlimits{\mathbb E}\limits_{#1} \left[ #2 \right]}
%\DeclareMathOperator*{\max}{max}
%\DeclareMathOperator*{\min}{min}
\DeclareMathOperator*{\argmax}{arg\,max}
\DeclareMathOperator*{\argmin}{arg\,min}
\DeclareMathOperator*{\E}{\mathbb{E}}


% Property env
% \newtheorem{property}{Propriété}[section]
% Boîte pour les DÉFINITIONS (Style bleu)
\newtcbtheorem[number within=chapter]{property}{Propriété}{
    enhanced,           % Permet des styles avancés
    colback=blue!5!white,   % Couleur de fond (bleu très clair)
    colframe=blue!75!black, % Couleur du cadre (bleu foncé)
    fonttitle=\bfseries,    % Titre en gras
    attach boxed title to top left={yshift*=-\tcboxedtitleheight/2, xshift=5mm}, % Titre "posé" sur le cadre
    boxed title style={colback=blue!75!black}
}{def}

% Proof env
\newtheorem{proof}{Preuve}[section]


% Boîte pour les DÉFINITIONS (Style bleu)
\newtcbtheorem[number within=chapter]{definition}{Définition}{
    enhanced,           % Permet des styles avancés
    colback=blue!5!white,   % Couleur de fond (bleu très clair)
    colframe=blue!75!black, % Couleur du cadre (bleu foncé)
    fonttitle=\bfseries,    % Titre en gras
    attach boxed title to top left={yshift*=-\tcboxedtitleheight/2, xshift=5mm}, % Titre "posé" sur le cadre
    boxed title style={colback=blue!75!black}
}{def}

% Boîte pour les THÉORÈMES (Style rouge/orange)
\newtcbtheorem[number within=chapter]{theorem}{Théorème}{
    enhanced,
    colback=red!5!white,
    colframe=red!75!black,
    fonttitle=\bfseries,
    attach boxed title to top left={yshift*=-\tcboxedtitleheight/2, xshift=5mm},
    boxed title style={colback=red!75!black}
}{thm}

% Boîte pour les LEMMES (Style vert)
\newtcbtheorem[number within=chapter]{lemma}{Lemme}{
    enhanced,
    colback=green!5!white,
    colframe=green!75!black,
    fonttitle=\bfseries,
    attach boxed title to top left={yshift*=-\tcboxedtitleheight/2, xshift=5mm},
    boxed title style={colback=green!75!black}
}{lem}


\newtcbtheorem[number within=chapter]{example}{Exemple}{
    enhanced,
    colback=gray!5!white,       % Fond gris très pâle (presque blanc)
    colframe=gray!70!black,     % Cadre gris anthracite (sobre)
    coltitle=white,             % Titre en blanc
    fonttitle=\bfseries,        % Titre en gras
    attach boxed title to top left={yshift=-2mm, xshift=2mm}, % Position du titre
    boxed title style={
        colback=gray!70!black,  % Fond du titre (identique au cadre)
        sharp corners           % Coins carrés
    },
    breakable,                  % Permet de couper l'exemple sur plusieurs pages
    terminator sign={ :},       % Ponctuation après le titre
    separator sign={~}          % Espace insécable
}{ex}



% --- BOÎTE GÉNÉRIQUE (Moteur interne) ---
\newtcolorbox{mediabox}[2][]{
    enhanced,
    width=0.85\textwidth,      % Largeur similaire à votre ancien userdefinedwidth
    center,                    % Centrer la boite
    colback=white,             % Fond blanc
    colframe=#2,               % Couleur du cadre
    sidebyside,                % Mode côte à côte (remplace tabular)
    sidebyside align=center,   % Alignement vertical
    lefthand width=1.5cm,      % Espace pour l'icône
    boxsep=5pt,                % Marges internes
    rounded corners,           % Coins arrondis
    arc=3mm,                   % Rayon des coins
    before skip=10pt,          % Espace AVANT la boite (remplace \medskip)
    after skip=10pt,           % Espace APRÈS la boite
    #1                         % Options supplémentaires
}

% --- COMMANDE VIDEO ---
% Utilisation dans le texte : \video{Titre}{URL} (inchangé)
\newcommand{\video}[2]{%
    \begin{mediabox}[colframe=red!75!black]{red!75!black}
        \centering
        \includegraphics[width=12mm]{fig/youtube.png} % Ajustez la taille si besoin
        \tcblower
        \textbf{\textsf{Capsule vidéo}} \\
        \textbf{\textit{#1}} \\
        \footnotesize{\url{#2}}
    \end{mediabox}
}

% --- COMMANDE COLAB ---
% Utilisation dans le texte : \colab{Titre}{URL} (inchangé)
\newcommand{\colab}[2]{%
    \begin{mediabox}[colframe=orange!80!black]{orange!80!black}
        \centering
        \includegraphics[width=12mm]{fig/colab.png} % Ajustez la taille si besoin
        \tcblower
        \textbf{\textsf{Exercice de code}} \\
        \textbf{\textit{#1}} \\
        \footnotesize{\url{#2}}
    \end{mediabox}
}

% --- ENVIRONNEMENT NOTE (Style unifié) ---
% On définit d'abord la boite tcolorbox
\newtcolorbox{notebox}[1]{
    enhanced,
    colback=teal!5!white,       % Fond très pâle (Bleu-Vert)
    colframe=teal!60!black,     % Cadre foncé
    coltitle=white,             % Titre en blanc
    fonttitle=\bfseries,
    title={#1},                 % Le titre vient de l'argument
    attach boxed title to top left={yshift=-2mm, xshift=2mm},
    boxed title style={
        colback=teal!60!black,
        sharp corners
    },
    breakable,                  % Permet de couper la note sur plusieurs pages
    before skip=10pt,
    after skip=10pt
}

% On redéfinit la commande \note pour qu'elle utilise cette boite
% Syntaxe : \note{Titre}{Contenu}
\newcommand{\note}[2]{%
    \begin{notebox}{#1}
        #2
    \end{notebox}
}


% Define a new column type for centered, fixed width, wrapped text
\newcolumntype{C}[1]{>{\centering\arraybackslash}p{#1}}
\newcolumntype{L}[1]{>{\raggedright\arraybackslash}p{#1}}


