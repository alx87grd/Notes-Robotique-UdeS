\chapter{Trigonométrie}


\subsection{Fonctions trigonométriques}
À venir!

%TODO graphiques sin cos 


\subsection{Identités trigonométriques}


Somme au carré:
%%%%%%%%%%%%%%%%%%%%%%%%%
\begin{align}
\sin^2(\theta) + \cos^2(\theta) = 1
\end{align}
%%%%%%%%%%%%%%%%%%%%%%%%%

Inversion du signe de l'angle:
%%%%%%%%%%%%%%%%%%%%%%%%%
\begin{align}
\sin\left(- \theta \right) &= - \sin\left( \theta \right) \\
\cos\left(- \theta \right) &=   \cos\left( \theta \right) %\\
%\tan\left(- \theta \right) &= - \tan\left( \theta \right)
\end{align}
%%%%%%%%%%%%%%%%%%%%%%%%%

Déphasage de 90$^o$:
%%%%%%%%%%%%%%%%%%%%%%%%%
\begin{align}
\sin\left(\frac{\pi}{2} - \theta \right) &= \cos(\theta) \\
\cos\left(\frac{\pi}{2} - \theta \right) &= \sin(\theta) 
\end{align}
%%%%%%%%%%%%%%%%%%%%%%%%%

Déphasage de 180$^o$:
%%%%%%%%%%%%%%%%%%%%%%%%%
\begin{align}
\sin\left(\pi - \theta \right) &= \sin(\theta) \\
\cos\left(\pi - \theta \right) &= -\cos(\theta) 
\end{align}
%%%%%%%%%%%%%%%%%%%%%%%%%

Somme de deux angles:
%%%%%%%%%%%%%%%%%%%%%%%%%
\begin{align}
\sin\left(\theta_1 + \theta_2 \right) = \sin( \theta_1 ) \cos( \theta_2 ) + \cos( \theta_1 ) \sin( \theta_2 )
\\
\sin\left(\theta_1 - \theta_2 \right) = \sin( \theta_1 ) \cos( \theta_2 ) - \cos( \theta_1 ) \sin( \theta_2 )
\\
\cos\left(\theta_1 + \theta_2 \right) = \cos( \theta_1 ) \cos( \theta_2 ) - \sin( \theta_1 ) \sin( \theta_2 )
\\
\cos\left(\theta_1 - \theta_2 \right) = \cos( \theta_1 ) \cos( \theta_2 ) + \sin( \theta_1 ) \sin( \theta_2 ) 
\end{align}
%%%%%%%%%%%%%%%%%%%%%%%%%

Dérivés:
%%%%%%%%%%%%%%%%%%%%%%%%%
\begin{align}
\frac{d}{d\theta} \, \sin\left(\theta \right) &= \cos(\theta) \\
\frac{d}{d\theta} \, \cos\left(\theta \right) &= -\sin(\theta) 
\end{align}
%%%%%%%%%%%%%%%%%%%%%%%%%

