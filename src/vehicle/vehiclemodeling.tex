\chapter{Modélisation des véhicules}

Ce chapitre propose un survol des approches de modélisation couramment utilisés dans un contexte de commande et de planification.


% ==========================================
% SECTION 1: THÉORIE UNIFIÉE
% ==========================================

\section{Introduction : L'art de la simplification}

Modéliser un véhicule, c'est l'art de mentir intelligemment. Un véhicule réel est un système infiniment complexe : déformation des pneus, turbulence de l'air, jeu dans les engrenages, vibrations du châssis. Pourtant, pour piloter un robot ou planifier un trajet, nous n'avons pas besoin de toute cette complexité. Ce chapitre explore comment nous passons de la réalité physique à des équations mathématiques exploitables par un ordinateur. L'objectif n'est pas de trouver le modèle parfait, mais le modèle suffisant pour une tâche à accomplir. Ici dans ce chapitre, l'accent est mis sur les modèles exploitables pour la synthèse de lois de commande ou la planification de trajectoire. Les approches de modélisation haute fidélité, telles que les éléments finis pour la déformation des pneus ou la mécanique des fluides numérique pour l'aérodynamisme complexe, généralement réservées à la validation en simulation, ne seront pas traitées ici.

\subsection{Le spectre de la fidélité}

Imaginez trois situations de conduite différentes. Dans chacune, votre cerveau utilise un modèle différent:

\begin{enumerate}
    \item \textbf{Navigation (GPS) :} Vous planifiez un trajet de Montréal à Québec. À cette échelle, votre voiture est un simple \textbf{point} qui suit une ligne. La largeur de la voiture, son angle de braquage ou la pression des pneus n'ont aucune importance.
    \item \textbf{Stationnement :} Vous vous garez dans une rue étroite. Ici, le modèle "point" ne suffit plus. Vous devez tenir compte de la \textbf{géométrie} : la voiture ne peut pas se déplacer latéralement (contrainte non-holonome), elle doit braquer et avancer.
    \item \textbf{Conduite sur glace :} Vous prenez un virage serré sur une chaussée glissante. La géométrie ne suffit plus. Il faut comprendre la \textbf{physique} : l'inertie, la friction, le transfert de masse. Si vous tournez le volant, la voiture ne va pas nécessairement suivre la direction des roues.
\end{enumerate}

En robotique, la trousse à outil comprend une hiérarchie de modèles (voir Tableau \ref{tab:genericvehicles}). Pour un algorithme de conduite autonome, il est parfois suffisant d'utiliser un modèle particule $(x,y)$ pour la planification de chemin à haut niveau, alors qu'un modèle dynamique précis sera nécessaire pour calculer des forces de commande en temps réel. Cet \textbf{écosystème de modèles est donc une boîte à outils}, où chaque modèle est un instrument adapté à une tâche spécifique.

%%%%%%%%%%%%%%%%%%%%%%%%%%%%%%%
\begin{figure}[htb]
	\centering
		\includegraphics[width=0.70\textwidth]{fig/holonomicvehicle.jpg}
	\caption{Modèle particule cinématique 2D pour un véhicule holonomique. La configuration est donnée par la position soit $\mathbf{q} = [x, y]^T \in \mathbb{R}^2$, l'état est identique à la configuration, et les actions sont les vitesses linéaires $\mathbf{u} = [v_x, v_y]^T \in \mathbb{R}^2$.}
	\label{fig:holonomicvehicle}
\end{figure}
%%%%%%%%%%%%%%%%%%%%%%%%%%%%%%%



%%%%%%%%%%%%%%%%%%%%%%%%%%%%%%%%%%%%%%%%%%%%%%%%%%%%%%%%%%%%%%%%%%%%%%%%%%%%%%%%%%%%%%%%%%%%%
\newpage
\section{Les trois ingrédients de base d'un modèle de véhicule}
%%%%%%%%%%%%%%%%%%%%%%%%%%%%%%%%%%%%%%%%%%%%%%%%%%%%%%%%%%%%%%%%%%%%%%%%%%%%%%%%%%%%%%%%%%%%%

Quel que soit le véhicule (drone, voiture, sous-marin), nous devons répondre à trois questions fondamentales pour le contrôler.

\subsection{La configuration $\mathbf{q}$: "Où suis-je ?"}

La configuration, notée $\mathbf{q}$, est l'ensemble minimal de variables nécessaires pour décrire la posture géométrique du véhicule à un instant figé. L'ensemble de toutes les configurations possibles forme l'espace de configuration, noté $\mathcal{C}$, dont la dimension correspond au nombre de degrés de liberté (DDL) du système.

La définition mathématique de $\mathbf{q}$ dépend du niveau d'abstraction choisi :
\begin{itemize}
    \item \textbf{Pour une particule (Point 2D/3D) :} La configuration se résume à la position cartésienne, soit $\mathbf{q} = [x, y]^T \in \mathbb{R}^2$ ou $\mathbf{q} = [x, y, z]^T \in \mathbb{R}^3$.
    \item \textbf{Pour un corps rigide 2D (Robot mobile) :} Il est nécessaire d'ajouter l'orientation (le cap) à la position, ce qui donne $\mathbf{q} = [x, y, \psi]^T$. Mathématiquement, cet espace est le groupe spécial euclidien $SE(2)$.
    \item \textbf{Pour un corps rigide 3D (Drone/Avion) :} La configuration inclut la position et l'attitude complète (souvent représentée par des angles d'Euler ou des quaternions), appartenant alors au groupe $SE(3)$.
\end{itemize}


\subsection{Les actions $\mathbf{u}$: "Que puis-je faire ?"}

Le vecteur d'action, noté $\mathbf{u}$, décrit les commandes ou les "entrées" que nous pouvons utiliser pour modifier le mouvement du véhicule. La nature de ces commandes change radicalement selon que l'on considère le véhicule comme un objet géométrique ou un système physique :
\begin{itemize}
    \item \textbf{Action Cinématique (Abstraite) :} On commande directement des vitesses (ex: "Avance à 1 m/s", "Tourne à 0.5 rad/s"). C'est l'hypothèse simplificatrice utilisée par la plupart des planificateurs de trajectoire. L'hypothèse sous-jacente est qu'il existe des contrôleurs de bas niveau suffisamment performants pour exécuter ces actions quasi-instantanément.
    \item \textbf{Action Dynamique (Réelle) :} On commande des efforts physiques (ex: "Couple moteur de 100 Nm", "Angle de braquage des roues", "Poussée des hélices"). C'est ce que le système de contrôle doit gérer en réalité à bas niveau.
\end{itemize}

Formellement, l'ensemble des entrées de commande admissibles est noté $\mathcal{U}$. Le vecteur $\mathbf{u} \in \mathcal{U}$ représente les variables d'entrée des équations du modèle utilisé.

\subsection{L'équation d'évolution : "Comment je bouge ?"}
C'est le cœur de la modélisation. C'est la fonction qui prédit l'évolution de la configuration en fonction des actions. On distingue deux grandes familles :

\paragraph{Modèles Cinématiques (Géométrie du mouvement)}
Ces modèles décrivent le mouvement possible sans s'occuper des forces. Ils répondent à la question : \textit{"Si j'avance en tournant avec telle vitesse, comment vont évaluer mes coordonnées ?"} Ils sont typiquement utiles à basse vitesse. Lorsque les actions sont des vitesses, ces équations prennent la forme suivante, i.e. des équations différentielles de premier ordre:
%%%%%%%%%%%%%%%%%%%%%%%%%%%%%%%%%%%%%%%
\begin{equation}
    \dot{\mathbf{q}} = \mathbf{N}(\mathbf{q})\mathbf{u}
\end{equation}
%%%%%%%%%%%%%%%%%%%%%%%%%%%%%%%%%%%%%%%

\paragraph{Modèles Dynamiques (Physique du mouvement)}
Ces modèles utilisent la seconde loi de Newton ($F=ma$). Ils répondent à la question : \textit{"Si j'applique telles forces, comment vont évoluer mes coordonnées?"}. Lorsque les actions sont des forces, ces équations prennent la forme d'équations différentielles d'ordre deux:
%%%%%%%%%%%%%%%%%%%%%%%%%%%%%%%%%%%%%%%
\begin{equation}
    \ddot{\mathbf{q}} = f_q(\dot{\mathbf{q}},\mathbf{q},\mathbf{u})
\end{equation}
%%%%%%%%%%%%%%%%%%%%%%%%%%%%%%%%%%%%%%%

Parfois, on s'intéresse seulement à l'évolution des vitesses d'un véhicule. On peut alors travailler avec des équations qui n'inclut que l'évolution des vitesses, sans prédire l'évolution de la configuration. Dans cette situation, on peut utiliser un vecteur de vitesses généralisées $\mathbf{\nu}$ et des équations différentielles d'ordre un:
%%%%%%%%%%%%%%%%%%%%%%%%%%%%%%%%%%%%%%%
\begin{equation}
    \dot{\mathbf{\nu}} = f_v(\dot{\mathbf{\nu}},\mathbf{u})
\end{equation}
%%%%%%%%%%%%%%%%%%%%%%%%%%%%%%%%%%%%%%%
Lorsque aucune propriétés ou forces ne dépend d'une variable configuration, on l'appelle parfois une coordonnée ignorable. On peut alors ignorer l'équation qui prédit d'élolution de cette variable et simplifier le système d'équations. Un exemple est la position $x$ d'un avion, il est possible de calculer l'évolution des vitesses de l'avion sans garder en mémoire l'évolution de sa position. Un contre-exemple est le tanguage d'un avion, l'angle d'attaque et donc la force de portance en dépende.

\paragraph{Formalisme d'état}

Pour un ordinateur, toutes ces notions sont unifiées sous la forme d'un système d'état. Le vecteur d'état, noté $\mathbf{x} \in \mathcal{X}$, est la "mémoire" du système nécessaire pour prédire le futur. L'ensemble $\mathcal{X}$ est l'espace d'état, et ses dimensions sont appelées l'ordre d'un système. Si $\mathcal{X}=\mathbb{R}^n$, on dira que le modèle est d'ordre $n$, ce qui correspond au nombres de variable indépendantes pour décrire l'état du système. 

\begin{itemize}
    \item \textbf{Pour un modèle cinématique}, l'état est simplement la configuration : $\mathbf{x} = \mathbf{q}$.
    \item \textbf{Pour un modèle dynamique}, l'état doit généralement inclure la configuration et les vitesses (linéaire et angulaire) : $\mathbf{x} = [\mathbf{q}, \dot{\mathbf{q}}]^T$. Sans connaître la vitesse actuelle, on ne peut pas prédire la position future.
\end{itemize}

L'équation générale de tout véhicule s'écrit alors sous la forme canonique :
%%%%%%%%%%%%%%%%%%%%%%%%%%%%%%%%%%%%%%%
\begin{equation}
    \dot{\mathbf{x}} = f(\mathbf{x}, \mathbf{u})
\end{equation}
%%%%%%%%%%%%%%%%%%%%%%%%%%%%%%%%%%%%%%%
C'est la forme standard pour résoudre des équations différentielles, typiquement utilisé pour les simulations sur un ordinateur.



% Insertion du Tableau 1 (Generic Vehicles) en mode paysage
\begin{landscape}
\subsection{Exemples de niveaux d'abstractions génériques pour véhicules}

Le tableau suivant donne quelques exemple de niveaux d'abstractions de véhicules généralement utilisé dans un contexte de commande, planification et analyse. Les modèles de particule sont typiquement utilisés pour planifier des chemins en fonction d'une carte d'obstacles. On peut utiliser une particule dynamique si on veut planifier le détail d'un profil de vitesse et des rayons de courbures réalisables, par exemple si on essaye d'optimiser une ligne de course sur un circuit. Ensuite, les modèles corps rigide cinématiques sont typiquement utilisés pour synthétisé des lois de commande pour des régimes à basse vitesse. C'est aussi le niveau d’abstraction qui fait apparaître le cap d'un véhicule et qui serait adapté pour planifier un stationnement parallèle par exemple. Finalement, les modèles dynamique de corps rigide sont utile pour les analyses de stabilité et synthétiser des lois de commande pour les drones, avions, etc. 



% ==========================================
% TABLE 1: GENERIC / HOLONOMIC MODELS (FR) - SIMPLIFIÉ
% ==========================================
\begin{table}[htbp]
    \centering
    \caption{\textbf{Abstractions de véhicules génériques}}
    \label{tab:genericvehicles}
    \vspace{0.1cm}
    \setlength{\tabcolsep}{4pt} % Espacement légèrement augmenté
    % Ajustement des largeurs : 1 col de desc + 5 cols de données
    \begin{tabular}{|L{0.09\linewidth}|L{0.17\linewidth}|L{0.17\linewidth}|L{0.17\linewidth}|L{0.17\linewidth}|L{0.17\linewidth}|}
        \hline
        \textbf{Caractér.} & 
        \textbf{Particule cinématique} (Ordre 1) & 
        \textbf{Particule dynamique} (Ordre 2) & 
        \textbf{Corps rigide cinématique 2D} (Ordre 1) & 
        \textbf{Corps rigide dynamique 2D} & 
        \textbf{Corps rigide dynamique 3D} \\
        \hline

         \textbf{n} & 
        2 & 
        4 & 
        3 & 
        6 & 
        12 
        \\
        \hline
        
        \textbf{Modèle dynamique} & 
        $\dot{x} = v_x$ 
        \newline 
        $\dot{y} = v_y$
        & 
        $\ddot{x} = a_x$ 
        \newline 
        $\ddot{y} = a_y$
        \newline 
         & 
        $\dot{\mathbf{p}} = \mathbf{R}(\psi)\mathbf{v}_{b}$ \newline $\dot{\psi} = r$  & 
        $m(\dot{u}-vr) = F_x$ \newline $m(\dot{v}+ur) = F_y$ \newline $I_z \dot{r} = M_z$ 
        & 
        $m(\dot{\mathbf{v}} + \omega \times \mathbf{v}) = \mathbf{F}$ \newline $\mathbf{I}\dot{\omega} + \omega \times \mathbf{I}\omega = \mathbf{M}$ \\
        \hline

        \textbf{Espace de config.} $\mathbf{q}$ & 
        $\mathbf{q} = [x, y]^T$ & 
        $\mathbf{q} = [x, y]^T$ & 
        $\mathbf{q} = [x, y, \psi]^T$ & 
        $\mathbf{q} = [x, y, \psi]^T$ & 
        $\mathbf{q} \in \mathbb{R}^{6}$ (Pos, Euler) \\ \hline
        
        \textbf{Espace d'état} $\mathbf{x}$ & 
        $\mathbf{x} = [x, y]^T$ & 
        $\mathbf{x} = [x, y, v_x, v_y]^T$ & 
        $\mathbf{x} = [x, y, \psi]^T$ & 
        $\mathbf{x} = [x, y, \psi, u, v, r]^T$ & 
        $\mathbf{x} \in \mathbb{R}^{12}$ \newline (Pos, Euler, VitLin, VitAng) \\
        \hline
        
        \textbf{Espace d'action} $\mathbf{u}$ & 
        $\mathbf{u} = [v_x, v_y]^T$ & 
        $\mathbf{u} = [a_x, a_y]^T$ & 
        $\mathbf{u} = [u, v, r]^T$ \newline (Vit. Corps) & 
        $\mathbf{u} = [F_x, F_y, M_z]^T$ \newline (Torseur 2D) & 
        $\mathbf{u} = [\mathbf{F}, \mathbf{M}]^T$ \newline (Torseur 3D) \\
        \hline
        
        % \textbf{Modèle Env. / Carte} & 
        % Carte traversabilité $trav(x,y) \in [0, 1]$ & 
        % Carte traversabilité $trav(x,y) \in [0, 1]$ \newline
        % Champ hauteur 2.5D \newline $z(x,y)$
        % & 
        % $trav(x,y) \in [0, 1]$ \newline
        % $z(x,y) \in \mathbb{R}^1$
        % & 
        % $trav(x,y) \in [0, 1]$ \newline
        % $z(x,y) \in \mathbb{R}^1$
        % & 
        % $trav(x,y) \in [0, 1]$ \newline
        % $z(x,y) \in \mathbb{R}^1$
        % \\
        % \hline
        
        \textbf{Contraintes d'état} & 
        Positions traversables \newline
        $\mathbf{q} \in \mathcal{C}_{free}$ & 
        Positions traversables \newline
        $\mathbf{q} \in \mathcal{C}_{free}$ \newline
        Vitesses réalisables \newline
        $\mathbf{u} \in \mathcal{V}_{feasible}$ & 
        $\mathbf{q} \in \mathcal{C}_{free} \subseteq \mathbb{R}^3$ 
        & 
        $\mathbf{q} \in \mathcal{C}_{free} \subseteq \mathbb{R}^3$ \newline
        $\mathbf{u} \in \mathcal{V}_{feasible} \subseteq \mathbb{R}^3$ 
        & 
        $\mathbf{q} \in \mathcal{C}_{free} \subseteq \mathbb{R}^6$ \newline
        $\mathbf{u} \in \mathcal{V}_{feasible} \subseteq \mathbb{R}^6$  
        \\
        \hline
        
        \textbf{Contraintes d'action} & 
        Vitesses réalisables \newline
        $\mathbf{u} \in \mathcal{V}_{feasible}$ & 
        Accélérations réalisables \newline
        $\mathbf{u} \in \mathcal{A}_{feasible}(x)$ & 
        $\mathbf{u} \in \mathcal{V}_{feasible} \subseteq \mathbb{R}^3$  & 
        $\mathbf{u} \in \mathcal{F}_{feasible}(x)$ & 
        $\mathbf{u} \in \mathcal{F}_{feasible}(x)$\\
        \hline
        
    \end{tabular}
\end{table}

\end{landscape}




\newpage
\section{Équations du mouvement dans le repère du corps}

Il est fréquent en robotique d'observer deux formulations dynamiques qui semblent distinctes : celle utilisée pour les bras manipulateurs (Partie I de ce cours) et celle utilisée pour les véhicules (Partie II). Il s'agit en fait de la même équation fondamentale, exprimée dans des systèmes de coordonnées différents.

\subsection{Coordonnées généralisées vs repère du corps}

\paragraph{Équation dans l'espace de configuration}
Il est toujours possible d'écrire les équations de la dynamique d'un système mécanique en utilisant uniquement l'espace de configuration, tel que présenté pour les robots manipulateurs \ref{sec:dynamic}. Pour un manipulateur, on exprime généralement la dynamique dans l'espace des coordonnées généralisées $\mathbf{q}$ (angles des joints). L'équation prend la forme standard :
\begin{equation}
    \mathbf{H}(\mathbf{q})\ddot{\mathbf{q}} + \mathbf{C}(\mathbf{q}, \dot{\mathbf{q}})\dot{\mathbf{q}} + \mathbf{g}(\mathbf{q}) = \boldsymbol{\tau}_q
    \label{eq:lagrangian_dynamics}
\end{equation}
Ici, la matrice d'inertie $\mathbf{H}(\mathbf{q})$ varie continuellement avec la configuration du robot (ex: lorsque le bras se déplie, l'inertie augmente). Il est théoriquement possible d'écrire la dynamique d'un véhicule sous cette forme, en utilisant les coordonnées généralisées appropriées (ex: position et orientation dans l'espace inertiel). Cependant, il est généralement plus pratique de formuler la dynamique des véhicules en utilisant des vitesses définies dans un repère attaché au corps du véhicule.

\paragraph{Repère du corps}
Pour un véhicule (drone, sous-marin), on préfère exprimer les vitesses $\boldsymbol{\nu}$ et les forces dans le repère attaché au corps, car les forces aérodynamiques/hydrodynamiques et les mesures des capteurs s'y expriment naturellement. L'équation devient :
\begin{align}
    \mathbf{M}_b \dot{\boldsymbol{\nu}} + \mathbf{C}_b(\boldsymbol{\nu})\boldsymbol{\nu} + \mathbf{g}_b(\mathbf{q}) &= \boldsymbol{\tau}_b \label{eq:newton_euler_dynamics}
    \\
    \mathbf{J}(\mathbf{q}) \dot{\mathbf{q}} &= \boldsymbol{\nu}
\end{align}
Pour un corps rigide, $\mathbf{J}(\mathbf{q})$ est la matrice jacobienne (ou de transformation cinématique) faisant le lien entre le repère inertiel et le repère du corps. En utilisant le repère attaché au corps, la matrice d'inertie $\mathbf{M}_b$ du véhicule est typiquement constante. Toutefois, on note l'apparition de termes de Coriolis et centrifuges dans $\mathbf{C}_b(\boldsymbol{\nu})$ qui n'étaient pas présents sous cette forme dans les coordonnées généralisées. Ces termes apparaissent car le repère du corps est un référentiel non-inertiel (en rotation).

\paragraph{Équivalence}
Le lien entre ces deux formulations s'établit par les relations suivantes :
\begin{align}
    \mathbf{H}(\mathbf{q}) &= \mathbf{J}^T(\mathbf{q}) \mathbf{M}_b \mathbf{J}(\mathbf{q}) \\
    \boldsymbol{\nu} &= \mathbf{J}(\mathbf{q}) \dot{\mathbf{q}}  \\
    \boldsymbol{\tau}_q &= \mathbf{J}^T(\mathbf{q}) \boldsymbol{\tau}_b
\end{align}
% On peut vérifier cette équivalence en comparant l'énergie cinétique $T$ exprimée dans les deux repères :
% \begin{align}
%     T = \frac{1}{2} \boldsymbol{\nu}^T \mathbf{M}_b \boldsymbol{\nu} 
%     = \frac{1}{2} (\mathbf{J} \dot{\mathbf{q}})^T \mathbf{M}_b (\mathbf{J} \dot{\mathbf{q}}) 
%     = \frac{1}{2} \dot{\mathbf{q}}^T (\underbrace{\mathbf{J}^T \mathbf{M}_b \mathbf{J}}_{\mathbf{H}(\mathbf{q})}) \dot{\mathbf{q}}
% \end{align}





%%%%%%%%%%%%%%%%%%%%%%%%%%%%%%%%%%%%%%%%%%%%%%%%%%%%%%%%%%%%%%%%%%%%%%%%%%%%%%%%%%%%%%%%%%%%%%%%%%%%%%%%%%%%%%%%%%%%%%%%%%%%%%%%%
\newpage
\subsection{Forme vectorielle et matricielle}
%%%%%%%%%%%%%%%%%%%%%%%%%%%%%%%%%%%%%%%%%%%%%%%%%%%%%%%%%%%%%%%%%%%%%%%%%%%%%%%%%%%%%%%%%%%%%%%%%%%%%%%%%%%%%%%%%%%%%%%%%%%%%%%%%

Les équations de la dynamique d'un corps rigide sont souvent présentées initialement sous forme vectorielle, séparant la translation et la rotation :
%%%%%%%%%%%%%%%%%%%%%%%%%%%%%%%%%%%%%%%%%
\begin{align}
        m (\dot{\mathbf{v}}_b + \boldsymbol{\omega}_b \times \mathbf{v}_b) &= \mathbf{F}_b \\
        \mathbf{I}_b \dot{\boldsymbol{\omega}}_b + \boldsymbol{\omega}_b \times (\mathbf{I}_b \boldsymbol{\omega}_b) &= \mathbf{M}_b
    \label{eq:newton_euler_vec}
\end{align}
%%%%%%%%%%%%%%%%%%%%%%%%%%%%%%%%%%%%%%%%%
Dans ces équations :
\begin{itemize}
    \item $m$ est la masse totale du véhicule.
    \item $\mathbf{I}_b$ est le tenseur d'inertie du véhicule exprimé dans le repère du corps (matrice $3 \times 3$ constante).
    \item $\mathbf{v}_b$ et $\boldsymbol{\omega}_b$ sont les vitesses linéaire et angulaire dans le repère du corps.
    \item $\mathbf{F}_b$ et $\mathbf{M}_b$ sont la force et le moment appliqués sur le corps.
\end{itemize}

Pour obtenir la forme matricielle unifiée, on définit les vecteurs de vitesses et d'effort généralisés :
%%%%%%%%%%%%%%%%%%%%%%%%%%%%%%%%%%%%%%%%%
\begin{equation}
    \boldsymbol{\nu} = \begin{bmatrix}
    \mathbf{v}_b \\
    \boldsymbol{\omega}_b
    \end{bmatrix}, \quad
    \boldsymbol{\tau}_b = \begin{bmatrix}
    \mathbf{F}_b \\
    \mathbf{M}_b
    \end{bmatrix}  
\end{equation}
%%%%%%%%%%%%%%%%%%%%%%%%%%%%%%%%%%%%%%%%%
En utilisant l'opérateur matriciel antisymétrique $S(\cdot)$ pour le produit vectoriel (tel que $\mathbf{a} \times \mathbf{b} = S(\mathbf{a})\mathbf{b}$), on peut réécrire les équations vectorielles (\ref{eq:newton_euler_vec}) ainsi :
%%%%%%%%%%%%%%%%%%%%%%%%%%%%%%%%%%%%%%%%%
\begin{align}
    m \dot{\mathbf{v}}_b + m S(\boldsymbol{\omega}_b) \mathbf{v}_b &= \mathbf{F}_b  \\
    \mathbf{I}_b \dot{\boldsymbol{\omega}}_b + S(\boldsymbol{\omega}_b) \mathbf{I}_b \boldsymbol{\omega}_b &= \mathbf{M}_b  
\label{eq:newton_euler_mat}
\end{align}
%%%%%%%%%%%%%%%%%%%%%%%%%%%%%%%%%%%%%%%%%
Ce système se regroupe finalement en une seule équation matricielle par blocs, correspondant à la forme (\ref{eq:newton_euler_dynamics}) :
\begin{equation}
    \underbrace{
    \begin{bmatrix}
        m \mathbf{I}_{3\times3} & \mathbf{0}_{3\times3} \\
        \mathbf{0}_{3\times3} & \mathbf{I}_b
    \end{bmatrix}
    }_{\mathbf{M}_b}
    \begin{bmatrix}
        \dot{\mathbf{v}}_b \\
        \dot{\boldsymbol{\omega}}_b
    \end{bmatrix}
    +
    \underbrace{
    \begin{bmatrix}
        m S(\boldsymbol{\omega}_b) & \mathbf{0}_{3\times3} \\
        \mathbf{0}_{3\times3} & -S(\mathbf{I}_b\boldsymbol{\omega}_b)
    \end{bmatrix}
    }_{\mathbf{C}_b(\boldsymbol{\nu})}
    \begin{bmatrix}
        \mathbf{v}_b \\
        \boldsymbol{\omega}_b
    \end{bmatrix}
    =
    \begin{bmatrix}
        \mathbf{F}_b \\
        \mathbf{M}_b
    \end{bmatrix}
\end{equation}

\subsection{Corps planaire (3 DDL)}
\label{sec:rigidbody2D}

Ce modèle s'applique aux véhicules dont le mouvement est restreint au plan. On définit les vitesses dans le repère du corps :
\begin{itemize}
    \item $u$ : Vitesse longitudinale (axe $x$).
    \item $v$ : Vitesse latérale (axe $y$).
    \item $r$ : Vitesse angulaire de lacet (rotation autour de $z$).
\end{itemize}

L'équation matricielle du mouvement $\mathbf{M}_b \dot{\boldsymbol{\nu}} + \mathbf{C}_b(\boldsymbol{\nu})\boldsymbol{\nu} = \boldsymbol{\tau}_b$ devient :
\begin{equation}
    \begin{bmatrix}
        m & 0 & 0 \\
        0 & m & 0 \\
        0 & 0 & I_z
    \end{bmatrix}
    \begin{bmatrix} \dot{u} \\ \dot{v} \\ \dot{r} \end{bmatrix}
    +
    \begin{bmatrix}
        0 & -m r & 0 \\
        m r & 0 & 0 \\
        0 & 0 & 0
    \end{bmatrix}
    \begin{bmatrix} u \\ v \\ r \end{bmatrix}
    =
    \begin{bmatrix} \sum F_x \\ \sum F_y \\ \sum M_z \end{bmatrix}
    \label{eq:planar_body_dynamics}
\end{equation}

\subsubsection{Équations scalaires}
En effectuant le produit matriciel, on obtient les équations scalaires. Les termes de droite représentent la somme de toutes les forces et moments externes (propulsion, aérodynamisme, contact pneu-sol, etc.) agissant sur le corps :
\begin{align}
    m(\dot{u} - rv) &= \sum F_x \\
    m(\dot{v} + ru) &= \sum F_y \\
    I_z \dot{r} &= \sum M_z
\end{align}

\subsection{Corps rigide (6 DDL)}

Ce modèle général s'applique aux véhicules évoluant dans l'espace 3D (ex: drones, avions, sous-marins). On définit les vitesses dans le repère du corps :
\begin{itemize}
    \item $u, v, w$ : Vitesses linéaires (Surge, Sway, Heave).
    \item $p, q, r$ : Vitesses angulaires (Roll, Pitch, Yaw rates).
\end{itemize}

Nous supposons ici que le repère du corps est aligné avec les axes principaux d'inertie, rendant la matrice d'inertie diagonale $\mathbf{I}_b = \text{diag}(I_x, I_y, I_z)$.

\subsubsection{Équations matricielles}
L'équation matricielle $\mathbf{M}_b \dot{\boldsymbol{\nu}} + \mathbf{C}_b(\boldsymbol{\nu})\boldsymbol{\nu} = \boldsymbol{\tau}_b$ s'écrit sous forme développée:
\begin{equation}
    \resizebox{0.95\hsize}{!}{$
    \begin{bmatrix}
        m & 0 & 0 & 0 & 0 & 0 \\
        0 & m & 0 & 0 & 0 & 0 \\
        0 & 0 & m & 0 & 0 & 0 \\
        0 & 0 & 0 & I_x & 0 & 0 \\
        0 & 0 & 0 & 0 & I_y & 0 \\
        0 & 0 & 0 & 0 & 0 & I_z
    \end{bmatrix}
    \begin{bmatrix} \dot{u} \\ \dot{v} \\ \dot{w} \\ \dot{p} \\ \dot{q} \\ \dot{r} \end{bmatrix}
    +
    \begin{bmatrix}
        0 & -mr & mq & 0 & 0 & 0 \\
        mr & 0 & -mp & 0 & 0 & 0 \\
        -mq & mp & 0 & 0 & 0 & 0 \\
        0 & 0 & 0 & 0 & I_z r & -I_y q \\
        0 & 0 & 0 & -I_z r & 0 & I_x p \\
        0 & 0 & 0 & I_y q & -I_x p & 0
    \end{bmatrix}
    \begin{bmatrix} u \\ v \\ w \\ p \\ q \\ r \end{bmatrix}
    =
    \begin{bmatrix} \sum F_x \\ \sum F_y \\ \sum F_z \\ \sum M_x \\ \sum M_y \\ \sum M_z \end{bmatrix}
    $}
\end{equation}

\subsubsection{Équations scalaires}
Le système matriciel ci-dessus correspond aux 6 équations scalaires d'Euler-Newton. Les termes de droite représentent la somme des forces et moments externes appliqués au corps :
\begin{align}
    m(\dot{u} - rv + qw) &= \sum F_x \\
    m(\dot{v} - pw + ru) &= \sum F_y \\
    m(\dot{w} - qu + pv) &= \sum F_z \\
    I_x \dot{p} + (I_z - I_y)qr &= \sum M_x \\
    I_y \dot{q} + (I_x - I_z)rp &= \sum M_y \\
    I_z \dot{r} + (I_y - I_x)pq &= \sum M_z
\end{align}

%%%%%%%%%%%%%%%%%%%%%%%%%%%%%%%%%%%%%%%%%%%%%%%%%%%%%%%%%%%%%%%%%%%%%%%%%%%%%%
\newpage
\subsection{Exemple avec un drone planaire}

Le drone est un cas d'école pertinent pour illustrer la différence entre les deux coordonnées de modélisation. Il s'agit d'un système sous-actionné possédant 3 degrés de liberté ($x, y, \theta$) mais seulement deux entrées de commande : un poussée $f_1$ et ne poussée $f_2$.
%%%%%%%%%%%%%%%%%%%%%%%%%%
\begin{figure}[htbp]
	\centering
		\includegraphics[width=0.99\textwidth]{fig/planar_drone.jpg}
	\caption{Modèle simplifié : Drone planaire (3-DOF)}
	\label{fig:planar_drone}
\end{figure}
%%%%%%%%%%%%%%%%%%%%%%%%%%

\paragraph{Approche A : Équations en coordonnées inertielles }

C'est l'approche la plus intuitive pour visualiser le mouvement. On écrit la seconde loi de Newton directement pour les coordonnées globales ($x, y$):
%%%%%%%%%%%%%%%%%%%%%%%%%%
\begin{equation}
    \underbrace{
    \begin{bmatrix}
        m & 0 & 0 \\
        0 & m & 0 \\
        0 & 0 & I
    \end{bmatrix}
    }_{\mathbf{H}(\mathbf{q})}
    \underbrace{
    \begin{bmatrix} \ddot{x} \\ \ddot{y} \\ \ddot{\theta} \end{bmatrix}
    }_{\ddot{\mathbf{q}}}
    +
    \underbrace{
    \begin{bmatrix} 0 \\ mg \\ 0 \end{bmatrix}
    }_{\mathbf{g}(\mathbf{q})}
    =
    \underbrace{
    \begin{bmatrix}
        -\sin(\theta) & -\sin(\theta) \\
        \cos(\theta) & \cos(\theta) \\
        -L & L
    \end{bmatrix}
    }_{\mathbf{B}(\mathbf{q})}
    \underbrace{
    \begin{bmatrix} f_1 \\ f_2 \end{bmatrix}
    }_{\mathbf{u}}
\end{equation}
%%%%%%%%%%%%%%%%%%%%%%%%%%

On remarque que la matrice d'inertie $\mathbf{H}$ est constante et diagonale (car exprimée dans le repère monde pour un corps symétrique simple), mais que la matrice d'actionneurs $\mathbf{B}(\mathbf{q})$ dépend fortement de l'orientation $\theta$, ce qui rend le système non-linéaire.


\paragraph{Approche B : Équations dans le repère du corps}

Si on utilise les vitesses exprimées dans le repère corps, l'équation équivalente sera:
%%%%%%%%%%%%%%%%%%%%%%%%%%
\begin{align}
    \underbrace{
    \begin{bmatrix}
        m & 0 & 0 \\
        0 & m & 0 \\
        0 & 0 & I
    \end{bmatrix}
    }_{\mathbf{M}_b}
    \underbrace{
    \begin{bmatrix} \dot{u} \\ \dot{v} \\ \dot{r} \end{bmatrix}
    }_{\dot{\mathbf{\nu}}}
    +
    \underbrace{
    \begin{bmatrix}
        0 & -mr & 0 \\
        mr & 0 & 0 \\
        0 & 0 & 0
    \end{bmatrix}
    }_{\mathbf{C}_b(\boldsymbol{\nu})}
    \underbrace{
    \begin{bmatrix} u \\ v \\ r \end{bmatrix}
    }_{\mathbf{\nu}}
    +
    \underbrace{
    \begin{bmatrix}
        mg \sin(\theta) \\
        mg \cos(\theta) \\
        0
    \end{bmatrix}
    }_{\mathbf{g}_b(\mathbf{q})}
    &=
    \underbrace{
    \begin{bmatrix}
        0 & 0 \\
        1 & 1 \\
        -L & L
    \end{bmatrix}
    }_{\mathbf{B}_b}
    \underbrace{
    \begin{bmatrix} f_1 \\ f_2 \end{bmatrix}
    }_{\mathbf{u}} \\
    \underbrace{
    \begin{bmatrix} \dot{x} \\ \dot{y} \\ \dot{\theta} \end{bmatrix}
    }_{\dot{\mathbf{q}}}
    &=
    \underbrace{
    \begin{bmatrix}
        \cos(\theta) & -\sin(\theta) & 0 \\
        \sin(\theta) & \cos(\theta) & 0 \\
        0 & 0 & 1
    \end{bmatrix}
    }_{J^{-1}}
    \underbrace{
    \begin{bmatrix} u \\ v \\ r \end{bmatrix}
    }_{\boldsymbol{\nu}}
\end{align}
%%%%%%%%%%%%%%%%%%%%%%%%%%
Comparé à l'approche A, on note trois différences fondamentales :
\begin{enumerate}
    \item La matrice d'allocation $\mathbf{B}_b$ est \textbf{constante} (elle ne dépend plus de $\theta$). 
    \item La matrice de Coriolis $\mathbf{C}_b(\boldsymbol{\nu})$ apparaît explicitement pour modéliser les forces virtuelles ressenties dans le repère tournant.
    \item La gravité $\mathbf{g}_b(\mathbf{q})$ est le seul terme qui dépend de l'orientation $\theta$. Elle représente la projection du vecteur poids dans le repère tournant du drone.
\end{enumerate}




\newpage
\section{Nomenclature et Notation}

Afin d'unifier la présentation des différents modèles (terrestres, marins, aériens) et de clarifier la distinction entre cinématique et dynamique, nous utiliserons la notation standard résumée dans le tableau \ref{tab:nomenclature}.

\begin{table}[htbp]
    \centering
    \caption{\textbf{Nomenclature des variables et symboles mathématiques}}
    \label{tab:nomenclature}
    \vspace{0.2cm}
    \renewcommand{\arraystretch}{1.5}
    \setlength{\tabcolsep}{5pt}
    \begin{tabular}{|c|l|p{0.50\linewidth}|}
        \hline
        \textbf{Symbole} & \textbf{Concept} & \textbf{Description} \\
        \hline
        \hline
        \multicolumn{3}{|c|}{\textbf{Espaces et Variables de Configuration}} \\
        \hline
        $\mathbb{R}^n$ & Espace Euclidien & Espace vectoriel réel de dimension $n$ (ex: position pure). \\
        \hline
        $SE(n)$ & Groupe Spécial Euclidien & Espace des mouvements rigides (Rotation + Translation). \\
        \hline
        $\mathbf{q}$ & Configuration & Posture géométrique instantanée (ex: $\mathbf{q} = [x, y, \theta]^T$). \\
        \hline
        $\dot{\mathbf{q}}, \ddot{\mathbf{q}}$ & Dérivées de config. & Vitesse et accélération généralisées dans l'espace de configuration (ex: $\dot{\mathbf{q}} = [\dot{x}, \dot{y}, \dot{\theta}]^T$). \\
        \hline
        \hline
        \multicolumn{3}{|c|}{\textbf{Cinématique (Repère du Corps)}} \\
        \hline
        $\mathbf{v}_b$ & Vitesse Linéaire & Vitesse de translation du CG exprimée dans le repère du corps (ex: $[u, v, w]^T$). \\
        \hline
        $\boldsymbol{\omega}_b$ & Vitesse Angulaire & Vitesse de rotation exprimée dans le repère du corps (ex: $[p, q, r]^T$). \\
        \hline
        $\boldsymbol{\nu}$ & Vitesse Généralisée & Vecteur regroupant vitesses linéaires et angulaires du corps : $\boldsymbol{\nu} = [\mathbf{v}_b^T, \boldsymbol{\omega}_b^T]^T$. \\
        \hline
        $\dot{\boldsymbol{\nu}}$ & Accélération du Corps & Dérivée temporelle des vitesses du corps. \\
        \hline
        $\mathbf{J}(\mathbf{q})$ & Matrice Jacobienne & Matrice de transformation liant les vitesses du corps aux dérivées de configuration : $\mathbf{J}(\mathbf{q}) \dot{\mathbf{q}} = \boldsymbol{\nu}$. \\
        \hline
        \hline
        \multicolumn{3}{|c|}{\textbf{Dynamique et Inertie}} \\
        \hline
        $m$ & Masse & Masse totale du véhicule. \\
        \hline
        $\mathbf{I}_b$ & Tenseur d'Inertie & Matrice $3 \times 3$ des moments d'inertie exprimée dans le repère du corps. \\
        \hline
        $\mathbf{M}_b$ & Matrice d'Inertie (Corps) & Matrice de masse généralisée ($6 \times 6$) constante dans le repère du corps (blocs $m\mathbf{I}_{3\times3}$ et $\mathbf{I}_b$). \\
        \hline
        $\mathbf{H}(\mathbf{q})$ & Matrice d'Inertie (Config) & Matrice d'inertie exprimée dans l'espace de configuration (dépend de $\mathbf{q}$). Liée par $\mathbf{H} = \mathbf{N}^{-T} \mathbf{M}_b \mathbf{N}^{-1}$. \\
        \hline
        $\boldsymbol{\tau}$ & Efforts Généralisés & Vecteur des forces et moments appliqués sur le corps. Ils peuvent être définit dans les coordonnées généralisée ou dans le repère du corps. \\
        \hline
    \end{tabular}
\end{table}




\newpage
% ==========================================
% SECTION 2: VÉHICULES TERRESTRES
% ==========================================
\section{Véhicules Terrestres (Automobiles et Robots mobiles)}

Cette section couvre les véhicules à roues! Le tableau \ref{tab:wheel_fidelity} présente une hiérarchie des modèles couramment utilisés pour les véhicules terrestres, allant des modèles les plus simples (particule) aux modèles les plus complexes (haute fidélité). Chaque niveau de modèle repose sur des hypothèses spécifiques et est adapté à des applications particulières, telles que la planification de trajectoire, le contrôle dynamique ou la simulation avancée.

\begin{table}[htbp]
    \centering
    \caption{\textbf{Hiérarchie de modèles pour véhicules à roues}}
    \label{tab:wheel_fidelity}
    \vspace{0.2cm}
    \renewcommand{\arraystretch}{1.5} % Aère les lignes pour la lisibilité
    \begin{tabular}{|p{0.26\linewidth}|p{0.30\linewidth}|p{0.30\linewidth}|}
        \hline
        \textbf{Type de modèle} & \textbf{Hypothèses principales} & \textbf{Utilité principale} \\
        \hline
        
        \textbf{Particule}  & 
        Véhicule considéré comme un point $(x,y)$& 
        Navigation globale (GPS) \\
        \hline
        
        \textbf{Bicyclette Cinématique}  & 
        Contrainte de roulement sans glissement (basse vitesse). Géométrie d'Ackermann. & 
        Planification de trajectoire locale, suivi de chemin à basse vitesse. \\
        \hline
        
        \textbf{Bicyclette Dynamique}  & 
        Mouvement planaire (3 DDL). Glissement des pneus modélisé. & 
        Analyse et commande de maneuvres dynamiques\\
        \hline
        
        \textbf{Véhicule Complet 3D} & 
        Corps rigide 6 DDL. 4 roues distinctes. Prise en compte du transfert de charge (roulis/tangage) et des suspensions. & 
        Analyse de stabilité, simulateur de conduites. \\
        \hline
        
        \textbf{Haute Fidélité}  & 
        Pneus et terrain déformables (FEA,DEM)  & 
        Simulation pour validations avancées. \\
        \hline
        
    \end{tabular}
\end{table}



\subsection{Modèle Bicyclette Cinématique}

Ces modèles reposent sur l'hypothèse fondamentale que les roues ne glissent pas (contrainte de roulement sans glissement). Ils sont valides à basse vitesse, lorsque les accélérations latérales sont négligeables (pas de dérapage).

\subsubsection{Architecture Ackermann et Simplification Bicyclette}

Pour qu'un véhicule à quatre roues puisse effectuer un virage sans qu'aucune roue ne dérape, les axes de rotation de toutes les roues doivent concourir en un point unique, appelé le \textbf{Centre Instantané de Rotation (CIR)}, voir Figure \ref{fig:ackerman}. Dans un véhicule classique, cela impose une contrainte géométrique appelée \textit{condition d'Ackermann} : lors d'un virage, la roue intérieure doit braquer davantage que la roue extérieure, car elle parcourt un cercle de rayon plus petit.

Le \textbf{modèle bicyclette} est une simplification qui consiste à regrouper les deux roues avant en une seule roue centrale virtuelle et les deux roues arrière en une seule roue arrière centrale. Cette approximation est valide tant que le véhicule est symétrique et que le transfert de charge latéral reste modéré (ce qui est cohérent avec l'hypothèse de basse vitesse).

\begin{figure}[htbp]
    \centering
        \includegraphics[width=0.90\textwidth]{fig/ackerman.jpg}
    \caption{Géométrie de direction d'Ackermann (gauche) et simplification bicyclette (droite). Notez que tous les axes convergent vers le CIR.}
    \label{fig:ackerman}
\end{figure}

\subsubsection{Équations cinématiques}

Nous cherchons à établir la relation entre les entrées de commande (vitesse longitudinale $u$ et angle de braquage $\delta$) et la variation de la posture du robot $\dot{\mathbf{q}} = [\dot{x}, \dot{y}, \dot{\theta}]^T$.

\paragraph{Choix du point de référence :}
Pour ce modèle cinématique simple, il est courant de définir la position $(x,y)$ du véhicule comme étant le \textbf{centre de l'essieu arrière}. Ce choix simplifie considérablement les équations car la vitesse latérale y est nulle (contrainte non-holonome). Si l'on désire un point de référence différents, il faut inclure des termes supplémentaires.

\textbf{Relation géométrique :}
En observant le triangle formé par le CIR, l'essieu arrière et l'essieu avant (voir Figure \ref{fig:bicyclemodel2}), on peut relier l'angle de braquage $\delta$ au rayon de courbure $R$ et à l'empattement $L$ (distance entre les essieux) :
\begin{equation}
    \tan(\delta) = \frac{L}{R} \quad \Rightarrow \quad R = \frac{L}{\tan(\delta)}
\end{equation}

\textbf{Relation cinématique :}
Considérons que la vitesse longitudinale au centre de l'essieu arrière est $u$ et que le véhicule possède une vitesse angulaire $\dot{\theta}$. Dans le repère du véhicule, le vecteur vitesse au niveau de l'essieu avant possède :
\begin{itemize}
    \item Une composante longitudinale égale à $u$.
    \item Une composante latérale égale à $L \dot{\theta}$ (due à la rotation du corps rigide autour de l'essieu arrière).
\end{itemize}
Le vecteur vitesse de la roue avant est donc $(u, L \dot{\theta})$. La contrainte de non-glissement impose que l'angle de braquage $\delta$ soit aligné avec ce vecteur vitesse :
\begin{equation}
    \tan(\delta) = \frac{L \dot{\theta}}{u} \quad \Rightarrow \quad \dot{\theta} = \frac{u}{L}\tan(\delta)
\end{equation}

\begin{figure}[htbp]
    \centering
        \includegraphics[width=0.90\textwidth]{fig/bicyclemodel2.jpg}
    \caption{Illustration des contraintes non-holonomes : la vitesse instantanée de chaque roue doit être alignée avec l'orientation des roues (pas de glissement latéral).}
    \label{fig:bicyclemodel2}
\end{figure}

\textbf{Modèle d'état :}
On peut maintenant formuler les équations différentielles liant les entrées de commande à la configuration. En utilisant le vecteur de configuration $\mathbf{q} = [x, y, \theta]^T$ (position et orientation dans le plan) et le vecteur de commande $\mathbf{u} = [u, \delta]^T$ (vitesse longitudinale et angle de braquage), on obtient les équations du mouvement suivantes :
\begin{align}
    \dot{x} &= u \cos(\theta) \\
    \dot{y} &= u \sin(\theta) \\
    \dot{\theta} &= \frac{u}{L} \tan(\delta)
\end{align}
Ces équations correspondent à un modèle d'état de la forme $\dot{\mathbf{q}} = f(\mathbf{q}, \mathbf{u})$, où le vecteur d'état est identique au vecteur de configuration, caractéristique des modèles purement cinématiques.

\begin{figure}[htbp]
    \centering
        \includegraphics[width=0.90\textwidth]{fig/bicyclemodel.jpg}
    \caption{Variables d'état du modèle bicyclette cinématique. $L$ est l'empattement et $\theta$ représente l'orientation du véhicule.}
    \label{fig:bicyclemodel}
\end{figure}



\subsection{Modèle Longitudinal}

\begin{figure}[htbp]
	\centering
		\includegraphics[width=0.90\textwidth]{fig/longitudinalmodel.jpg}
	\caption{Forces longitudinales s'appliquant au véhicule}
	\label{fig:longitudinalmodel}
\end{figure}


\subsection{Modèle Bicyclette Dynamique}

Le modèle bicyclette dynamique prend en compte les forces, les masses et les inerties dans le plan avec 3 DDL. On réutilise les équations de corps rigide (voir section \ref{sec:rigidbody2D}) avec les vitesses exprimées dans le repère du corps : $u$ (longitudinale), $v$ (latérale) et $r$ (vitesse angulaire de lacet), et on ajoute  les forces des pneus et une traînée aérodynamique:
%%%%%%%%%%%%%%%%%%%%%%%%%%%%%%%%%%%%%
\begin{align}
    m(\dot{u} - rv) &= (F_{x1} c_\delta - F_{y1} s_\delta) + F_{x2} - F_{aero} \\
    m(\dot{v} + ru) &= (F_{x1} s_\delta + F_{y1} c_\delta) + F_{y2} \\
    I_z \dot{r} &= a(F_{x1} s_\delta + F_{y1} c_\delta) - b F_{y2}
\end{align}
%%%%%%%%%%%%%%%%%%%%%%%%%%%%%%%%%%%%%
ou, tel qu'illustré à la figure \ref{fig:dynamic_bicycle} :
\begin{itemize}
    \item Indice $1$ : Roue avant (front), Indice $2$ : Roue arrière (rear).
    \item $a, b$ : Distances du centre de gravité (CG) aux essieux avant et arrière.
    \item $c_\delta, s_\delta$ : Notation compacte pour $\cos(\delta)$ et $\sin(\delta)$, où $\delta$ est l'angle de braquage.
    \item $F_{xi}, F_{yi}$ : Forces exprimées \textbf{dans le repère de la roue} (alignées avec le pneu).
    \item $F_{aero} = \frac{1}{2} \rho C_d A u |u|$ : Traînée aérodynamique s'opposant au mouvement.
\end{itemize}

\colab{Démonstration modèle bicyclette dynamique}{https://colab.research.google.com/drive/1NHcn2yDk9K5yCRhXo7X1MNb1Sp-bSHZh?usp=sharing}


\begin{figure}[htbp]
    \centering
        \includegraphics[width=0.90\textwidth]{fig/dynamicbicycle.jpg}
    \caption{Diagramme du corps libre. Les forces $F_{x,y}$ sont définies dans le repère de chaque roue.}
    \label{fig:dynamic_bicycle}
\end{figure}

% \begin{figure}[htbp]
%     \centering
%         \includegraphics[width=0.90\textwidth]{fig/dynamic_bicycle.jpg}
%     \caption{Cinématique de la roue : définition des vitesses locales et des angles de glissement.}
%     \label{fig:wheel_slip}
% \end{figure}

\subsubsection{Cinématique : Vitesses locales aux roues}

Les modèles de pneus sont basés sur des vitesses relatives à la roue, nous devons d'abord exprimer la vitesse du centre de chaque roue \textit{dans le repère de cette roue}.

\paragraph{Vitesses aux essieux (Repère Châssis)}
La cinématique du corps rigide nous donne la vitesse au point d'attache des essieux exprimée dans le repère du véhicule ($b$) :
\begin{align}
    \mathbf{v}_{1}^b &= [u, \quad v + a r]^T \\
    \mathbf{v}_{2}^b &= [u, \quad v - b r]^T
\end{align}

\paragraph{Projection dans le repère de la roue}
La roue avant ici n'a pas de rotation relative au corp, mais la roue avant est braquée de $\delta$. On applique une rotation inverse pour exprimé les vitesses dans son repère local:
\begin{align}
    % v_{x1}^c &= u \cos\delta + (v + ar)\sin\delta \\
    % v_{y1}^c &= -u \sin\delta + (v + ar)\cos\delta
    \mathbf{v}_{1}^c &= [u \cos\delta + (v + ar)\sin\delta, \quad -u \sin\delta + (v + ar)\cos\delta]^T
\end{align}

\subsubsection{Calcul des Glissements}

Les modèles de pneus sont souvent défini en terme de variable de taux de glissement. Les définitions exactes peuvent varier selon le contexte. 

\paragraph{Angle de dérive latéral ($\alpha$)}
Défini par le ratio entre vitesse latérale et longitudinale dans le repère de la roue. Le signe négatif est une convention pour assurer que la force latérale s'oppose au glissement.
%%%%%%%%%%%%%%%%%%%%%%%%
\begin{align}
    \alpha_1 &= - \arctan\left( \frac{v_{y1}^c}{|v_{x1}^c| + \epsilon} \right)  \approx \delta - \arctan\left( \frac{v_{y1}^b}{|v_{x1}^b| + \epsilon}\right) \\
    \alpha_2 &= - \arctan\left( \frac{v_{y2}^b}{|v_{x2}^b| + \epsilon} \right)
\end{align}
%%%%%%%%%%%%%%%%%%%%%%%%
\note{Note:}{Nous avons inclus un $epsilon$ au dénominateur dans les équations, ce qui est généralement inclus dans les méthodes numérique pour éviter les instabilités du modèle autour de vitesse nulles.}

\paragraph{Glissement longitudinal ($\kappa$)}
Défini par la différence normalisée entre la vitesse tangentielle de la roue ($R\omega$) et la vitesse au sol ($v_{x}^w$).
%%%%%%%%%%%%%%%%%%%%%%%%
\begin{align}
    \kappa_1 = \frac{R_1 \omega_1 - v_{x1}^c}{|v_{x1}^c| + \epsilon}
    \quad \quad 
    \kappa_2 = \frac{R_2 \omega_2 - v_{x2}^b}{|v_{x2}^b| + \epsilon}
\end{align}
%%%%%%%%%%%%%%%%%%%%%%%%

\begin{figure}[h]
    \centering
    \includegraphics[width=0.3\textwidth]{fig/wheel3D.jpg}
    \includegraphics[width=0.25\textwidth]{fig/wheelxy.jpg}
    \includegraphics[width=0.25\textwidth]{fig/wheelxz.jpg}
    \label{fig:wheelslip}
    \caption{Axes, vitesses et angles pour les modèles de pneus ($v_y<0$)}
\end{figure}



\subsubsection{Modèles de forces}

Une fois $\alpha$ et $\kappa$ calculés, la plupart des modèles de pneus ont la forme suivante:
%%%%%%%%%%%%%%%%%%%%%%%%
\begin{align}
    F_x = f(\alpha, \kappa, F_z) 
    \quad \quad
    F_y = f(\alpha, \kappa, F_z) 
\end{align}
%%%%%%%%%%%%%%%%%%%%%%%%

\begin{figure}[h]
    \centering
    \includegraphics[width=0.4\textwidth]{fig/ckappa.jpg}
    \includegraphics[width=0.5\textwidth]{fig/calpha.jpg}
    \label{fig:typicaltire}
    \caption{Courbes typique de pneus}
\end{figure}



\paragraph{Modèle Linéaire }
Lorsque le glissement est modéré, une modèle linéaire peut faire des bonnes prédictions:
%%%%%%%%%%%%%%%%%%%%%%%%
\begin{align}
    F_{x} &= C_k \kappa \\
    F_{y} &= C_\alpha \alpha
\end{align}
%%%%%%%%%%%%%%%%%%%%%%%%

Les paramètres $C$ sont souvent appelé les paramètres de rigidité des pneus, $C_\alpha$ est typiquement appelé \textit{cornering stiffness} et est un paramètre important pour les caractéristiques de maneuvrabilité d'un véhicule. Pour le modèle linéaire de base, les forces $x$ et $y$ sont indépendantes, toutefois il est courant d'ajouter une saturation sur ces forces basée sur le \textbf{"Cercle de Friction"}, pour limiter la force totale du pneu à une limite supérieur disponible $\mu F_z$ :


\paragraph{Modèle de Pacejka}
Surnommé dans le domaine \textbf{\textit{Magic Formula}}, ce modèle empirique capture le comportement non-linéaire et la saturation douce du pneu. La forme utilisée pour la force longitudinale ($F_x$) et latérale ($F_y$) est :
\begin{equation}
    F(x) = D \sin(C \arctan(B x - E(B x - \arctan(B x))))
\end{equation}
Où $x$ est le glissement ($\alpha$ ou $\kappa$). Les coefficients contrôlent la forme de la courbe :
\begin{itemize}
    \item $B$ (Stiffness) : Pente à l'origine (relié à $C_\alpha$).
    \item $C$ (Shape) : Forme de l'asymptote.
    \item $D$ (Peak) : Valeur maximale (relié à $\mu F_z$).
    \item $E$ (Curvature) : Courbure avant le pic.
\end{itemize}





\newpage
\subsection{Dynamique de Suspensions}
Lorsque le véhicule accélère ou tourne, le poids se transfère, modifiant la traction disponible.



\subsubsection{Modèle quart-de-véhicule}

\begin{figure}[htbp]
	\centering
		\includegraphics[width=0.90\textwidth]{fig/simplesuspension.jpg}
	\caption{Modèle de suspension 1-DDL (Quart de véhicule)}
	\label{fig:simplesuspension}
\end{figure}


\newpage
\subsection{Interaction Pneu-Sol et terramecanique}



\subsubsection{Modèles de Sol Bekker-Wong}
En robotique tout-terrain (agricole, planétaire), le sol se déforme sous la roue.













% ==========================================
% SECTION 3: VÉHICULES AÉRIENS
% ==========================================
\newpage
\section{Drones (Multirotors)}


\subsubsection{Modèle Planaire (2D)}
Utile pour introduire les concepts sans la complexité de la 3D.

\begin{figure}[htbp]
	\centering
		\includegraphics[width=0.99\textwidth]{fig/planar_drone.jpg}
	\caption{Modèle simplifié : Drone planaire (3-DOF)}
	\label{fig:planar_drone}
\end{figure}



% ==========================================
% SECTION 4: AVIONS
% ==========================================
\newpage
\section{Avions (Ailes fixes)}


\subsection{Modèle d'avion 3DDL (Plan vertical)}

Section en construction!

Cette section aborde la modélisation d'un aéronef à ailes fixes, dans le plan vertical (vue de côté). Le modèle à 3 DDL permet d'étudier la dynamique longitudinale :
\begin{itemize}
    \item \textbf{Surge ($u$)} : Vitesse longitudinale (axe du fuselage).
    \item \textbf{Heave ($v$)} : Vitesse verticale (axe perpendiculaire aux ailes).
    \item \textbf{Pitch ($r$)} : Vitesse de tangage (rotation du nez).
\end{itemize}

%%%%%%%%%%%%%%%%%%%%%%%%
\begin{figure}[htbp]
	\centering
		\includegraphics[width=0.99\textwidth]{fig/plane_2D.jpg}
	\caption{Modèle simplifié : Avion 2D}
	\label{fig:plane_2D}
\end{figure}
%%%%%%%%%%%%%%%%%%%%%%%%

Ce modèle capture les dynamiques essentielles de décrochage et de stabilité.

\colab{Modèle avion planaire}{https://colab.research.google.com/drive/15RBFDTUYSq0dFLBRd0VKLxJxXOONSYJI?usp=sharing}

\subsection{Équations du mouvement}

On peut utiliser les équations génériques du corps rigide planaire, mais en incluant les forces aérodynamique et la gravité. Une particularité du modèle d'avion, est qu'une des actions est une position, l'angle de la gouverne, donc pas une source de force. Dans les équations, cela ce reflète comme un dépendance directe du vecteur de force aérodynamique à cette entrée
%%%%%%%%%%%%%%%%%%%%%%%%%%%%%%
\begin{equation}
    \mathbf{M} \dot{\boldsymbol{\nu}} + \mathbf{C}(\boldsymbol{\nu})\boldsymbol{\nu} + \mathbf{g}(\mathbf{q}) + \mathbf{f}_{aero}(\boldsymbol{\nu}, \delta) = \mathbf{B} \mathbf{f_p}
\end{equation}
%%%%%%%%%%%%%%%%%%%%%%%%%%%%%%0
où :
\begin{itemize}
    \item $\boldsymbol{\nu} = [u, v, r]^T$ est le vecteur vitesse dans le repère du corps.
    \item $\mathbf{g}(\mathbf{q})$ est le vecteur gravité projeté dans le repère du corps (voir section \ref{sec:rigidbody2D}).
    \item $\mathbf{f}_{aero}(\boldsymbol{\nu}, \delta)$ représente les forces de portance, traînée et moment.
    \item $\mathbf{B} \mathbf{f_p}$ représente les forces des propulseurs.
\end{itemize}

\subsubsection{Forces Aérodynamiques}

Dans notre modèle simplifié, deux forces aérodynamiques s'appliquent sur le véhicule aux \textbf{centres de poussée} (CP) de l'aile et de la queue:
\begin{enumerate}
    \item L'aile principale, situé à une distance $l_{w}$ arrière du CG.
    \item Queue, situé à une distance $l_{t}$ arrière du CG.
\end{enumerate}

\paragraph{Forces dans le repère vent }
La convention pour définir des forces aérodynamiques est de se placer dans un repère aligné avec le mouvement, la composante aligné avec la vitesse relative au fluide est appelée la trainée et la composante perpendidulaire est appelée la portance.  L'air génère des forces que l'on approxime comme proportionnelles au carré de la vitesse relative absolue $V = \sqrt{u^2+v^2}$, avec des coefficients dépendant de l'angle d'attaque $\alpha$ :
\begin{equation}
    \alpha = \arctan\left(\frac{v}{u}\right)
\end{equation}

Les forces aérodynamiques sont définies par les coefficients de portance ($C_L$), de traînée ($C_D$) et de moment ($C_m$):
\begin{equation}
    \mathbf{f}^c_{aero}(\alpha) 
    = 
    \begin{bmatrix}
        -\frac{1}{2} \rho S C_D(\alpha) V^2 \\
        \frac{1}{2} \rho S C_L(\alpha)  V^2 \\
        \frac{1}{2} \rho S \bar{c} C_m(\alpha)  V^2
    \end{bmatrix}
\end{equation}
Notez que la traînée ($C_D$) s'oppose au mouvement (signe négatif dans la direction $\hat{c}_1$ et la portance ($C_L$) agit perpendiculairement (axe $\hat{c}_2$).

Ici on utilise un modèle simplifié "tout régime" (incluant le vol normal et les acrobaties/décrochage), avec des courbes caractéristiques d'une aile (profil symétrique) par :
%%%%%%%%%%%%%%%%%%%%%%%%%%%%%%
\begin{align}
    C_L(\alpha) &= C_{L,max} \sin(2\alpha) \\
    C_D(\alpha) &= C_{D,0} + C_{D,max} \sin^2(\alpha) \\
    C_m(\alpha) &= -C_{m,max} \sin(\alpha) \cos(\alpha)
\end{align}
%%%%%%%%%%%%%%%%%%%%%%%%%%%%%%

On projette ensuite ces forces dans le repère du corps à l'aide d'une matrice de rotation dépendant de l'angle d'attaque $\alpha$. L'équation de transformation est :
\begin{equation}
    \mathbf{f}^b_{aero} = 
    \begin{bmatrix}
        \cos\alpha & -\sin\alpha & 0 \\
        \sin\alpha & \cos\alpha & 0 \\
        0 & 0 & 1
    \end{bmatrix}
    \mathbf{f}^c_{aero}
\end{equation}


\paragraph{Somme des forces et moments au CG}
Le force aérodynamique générale $\mathbf{f}_{aero}(\boldsymbol{\nu}, \delta)$ appliqué au centre de gravité est la somme des contributions de chaque surface.

TODO: détails à venir

\subsection{Propulsion et Vecteur de Commande}

Le vecteur de commande est $\mathbf{u} = [f_p, \delta_e]^T$. Ici nous avons une force (propulsion) et une configuration géométrique (angle $\delta_e$).

\begin{enumerate}
    \item \textbf{Poussée ($f_p$)} : Force générée par le moteur, appliquée généralement selon l'axe longitudinal $x$ du corps. Elle entre dans l'équation via la matrice $\mathbf{B}$.
    \item \textbf{Élévateur ($\delta$)} : Comme vu précédemment, il n'apparaît pas dans la matrice $\mathbf{B}$ linéaire classique, mais modifie directement le vecteur $\mathbf{f}_{aero}$.
\end{enumerate}

\begin{equation}
    \mathbf{B}\mathbf{f}_p = 
    \begin{bmatrix}
        1 \\ 0 \\ 0
    \end{bmatrix} f_p
\end{equation}


% ==========================================
% SECTION 4: AVIONS
% ==========================================
\newpage
\section{Véhicules nautiques}


\subsection{Modèle de manœuvre basse-vitesse}

Section en construction!

Cette section s'intéresse à la modélisation des navires et robots de surface. Bien qu'un navire évolue dans un espace 3D (6 DDL), pour les applications de positionnement dynamique ou de manœuvre de port, on utilise couramment un modèle planaire simplifié à 3 DDL :
\begin{itemize}
    \item \textbf{Surge ($u$)} : Avancement longitudinal.
    \item \textbf{Sway ($v$)} : Dérive latérale.
    \item \textbf{Yaw ($r$)} : Lacet (cap).
\end{itemize}

%%%%%%%%%%%%%%%%%%%%%%%%
\begin{figure}[htbp]
	\centering
		\includegraphics[width=0.99\textwidth]{fig/boat_2D.jpg}
	\caption{Modèle simplifié : Bateau 2D}
	\label{fig:boat_2D}
\end{figure}
%%%%%%%%%%%%%%%%%%%%%%%%

Le modèle présenté ici est basé sur le livre de Fossen \cite{fossen2021handbook}, spécifiquement les modèles de manœuvre à basse vitesse.

\colab{Modèle de véhicule nautique planaire basse vitesse}{LINK_TODO}


\subsection{Équations du mouvement}

L'équation du mouvement dans le repère du corps suit la structure standard Newton-Euler, mais adaptée pour inclure les effets hydrodynamiques :
\begin{equation}
    \mathbf{M} \dot{\boldsymbol{\nu}} + \mathbf{C}(\boldsymbol{\nu})\boldsymbol{\nu} + \mathbf{d}(\boldsymbol{\nu}) = \mathbf{B} \mathbf{u}
\end{equation}
où :
\begin{itemize}
    \item $\boldsymbol{\nu} = [u, v, r]^T$ est le vecteur vitesse dans le repère du corps.
    \item $\mathbf{M}$ est la matrice d'inertie (incluant théoriquement la masse ajoutée).
    \item $\mathbf{d}(\boldsymbol{\nu})$ est les forces hydrodynamiques (traînée).
    \item $\mathbf{B} \mathbf{u}$ représente les forces des propulseurs.
\end{itemize}

\subsubsection{Forces Hydrodynamiques}
L'eau oppose une résistance complexe au mouvement. Le modèle implémenté combine deux effets, un amortissement visqueux linéaire qui domine à très basse vitesse et des lois de portances/trainées quadratique, similaires aux modèles aérodynamique des avions. L'approche utilisée calcule les coefficients de traînée $C_x, C_y, C_m$ en fonction de l'angle d'incidence avec l'eau :
\begin{equation}
    \alpha = -\arctan\left(\frac{v}{u}\right)
\end{equation}
Les forces sont ensuite calculées selon la pression dynamique :
\begin{equation}
    \mathbf{d}_{quad}(\boldsymbol{\nu}) 
    =
    \begin{bmatrix}
        -\frac{1}{2} \rho A_{front} C_x(\alpha) V^2 \\[6pt]
        -\frac{1}{2} \rho A_{lat} C_y(\alpha) V^2 \\[6pt]
        -\frac{1}{2} \rho A_{lat} L_{oa} C_m(\alpha) V^2
    \end{bmatrix}
\end{equation}
où $V^2 = u^2 + v^2$ et les $C_x,C_y,C_m$ sont des courbes de coefficients fonction de l'angle avec le courant. Dans un modèle simplifié (basé sur \cite{fossen2021handbook}), ces coefficients sont approximés par des fonctions trigonométriques:

%%%%%%%%%%%%%%%%%%%%%%%%%%%%%%
\begin{align}
    C_x(\alpha) &= -C_{x,max} \cos(\alpha) |\cos(\alpha)| \\
    C_y(\alpha) &= C_{y,max} \sin(\alpha) |\sin(\alpha)| \\
    C_m(\alpha) &= C_{m,max} \sin(2\alpha)
\end{align}
%%%%%%%%%%%%%%%%%%%%%%%%%%%%%%

\subsection{Propulsion}

Si la propulsion est un propulseur orientable situé à l'arrière du bateau, la force de proulsion est un vecteur de force 2D $\mathbf{u} = [T_x, T_y]^T$ appliqué à une distance $l_t$ derrière le centre de gravité (par exemple, un moteur hors-bord ou un safran orientable).

La relation entre la commande et les efforts généralisés sur le corps est donnée par la matrice d'allocation $\mathbf{B}$ :
%%%%%%%%%%%%%%%%%%%%%%%%%%%%%%
\begin{equation}
    \boldsymbol{\tau} = \mathbf{B} \mathbf{u} = 
    \begin{bmatrix}
        1 & 0 \\
        0 & 1 \\
        0 & -l_t
    \end{bmatrix}
    \begin{bmatrix} T_x \\ T_y \end{bmatrix}
    =
    \begin{bmatrix} T_x \\ T_y \\ -l_t T_y \end{bmatrix}
\end{equation}
%%%%%%%%%%%%%%%%%%%%%%%%%%%%%%
On remarque qu'une force latérale $T_y$ génère à la fois une force de dérive et un moment de lacet, créant un couplage fort typique des navires.