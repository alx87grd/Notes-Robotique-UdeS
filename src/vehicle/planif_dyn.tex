\chapter{Optimisation de trajectoires}
\label{chap:trajopt}

Sources utille:\\
\url{https://arxiv.org/pdf/1707.00284}\\
\url{https://underactuated.mit.edu/trajopt.html}

Chapitre en construction!!

\video{Introduction à la commande optimale}{https://youtu.be/3x6Vg-RRZ50}

\colab{Démo d'introduction à l'optimisation de trajectoires}{https://colab.research.google.com/drive/1yq2GHAkvO6fTF2W-tRbACDBa9_scec2k?usp=sharing}


\section{Formulation et principes de résolution}

L'optimisation de trajectoire consiste à trouver une trajectoire $\{x^{*}(t),u^{*}(t)\}$ qui minimise une fonction de coût tout en satisfaisant la dynamique du système et des contraintes du système. Contrairement à la programmation dynamique qui cherche une politique globale, cette approche produit une solution locale pour une condition initiale donnée, ce qui permet de traiter des systèmes de grande dimension.

Une fois discrétisé, le problème est converti en un problème de \textbf{programmation non-linéaire (NLP)} sous la forme standard :
\begin{align*}
    \min_{z} \quad & J(z) \\
    \text{sujet à} \quad & \col{l} \leq \col{c}(z) \leq \col{u}
\end{align*}

Pour résoudre ce NLP, on fait appel à des solveurs numériques spécialisés tels que \textbf{SNOPT} (algorithme SQP pour grandes échelles), \textbf{IPOPT} (méthode de points intérieurs robuste) ou \textbf{FMINCON}.

\begin{definition}{Problème de commande optimale (Forme de Bolza)}{def:bolza}
    Le problème consiste à minimiser le coût $J$ défini par :
    $$ J = \phi(t_0, x_0, t_f, x_f) + \int_{t_0}^{t_f} g(t, x, u) dt $$
    Sujet à :
    \begin{itemize}
        \item \textbf{Dynamique :} $\dot{x} = f(t, x, u)$
        \item \textbf{Contraintes :} $c_{min} < c(t, x, u) < c_{max}$
        \item \textbf{Conditions limites :} $b_{min} < b(t_0, x_0, t_f, x_f) < b_{max}$
    \end{itemize}
\end{definition}



\section{Planéité différentielle}
\label{sec:diff_flatness}

La planéité différentielle (\textit{differential flatness}) est une propriété permettant de simplifier drastiquement la planification pour certains systèmes sous-actionnés (ex: drones, voitures).

\begin{definition}{Planéité différentielle}{def:diff_flatness}
    Un système est différentiellement plat s'il existe des sorties "plates" telles que tous les états et toutes les commandes peuvent être calculés uniquement à partir de ces sorties et de leurs dérivées temporelles.
\end{definition}

Le concept de planéité différentielle, en anglais \textit{differential flatness}, peut être très utile pour planifier des trajectoires pour des systèmes sous-actionnés. Le principe est le suivant: certains systèmes ont des variables sorties qui ont une propriété spéciale, i.e. nommée la planéité différentielle, qui est que tout les états et forces du système sont entièrement définis par ces variables et leurs dérivées supérieures. Par exemple, la position de l'effecteur pour un bras manipulateur, le centre de masse d'un drone quadcopter, le point milieu entre les deux roues arrières d'une voiture ont cette propriété. Pour une bras manipulateur pleinement actionné, les forces nécessaires aux joints sont reliés à l'accélération à l'effecteur. Pour un drone, les forces de propulsion sont reliés à la quatrième dérivée temporelle de la position du centre de masse, souvent appelé le \textit{snap}.

Ce concept peut être exploité pour générer plus facilement des trajectoires faisable qui respecte la dynamique du système. L'idée est de définir la trajectoire désirée du système en définissant une trajectoire continue juste qu'à un ordre $n$ (le minimum est variable selon le système), qui permet d'ensuite calculer tous les autres états et forces à partir de cet trajectoire et les dérivées supérieures de celle-ci. Typiquement, on peut utiliser une paramétrisation de type polynomiale d'ordre $n$ pour garantir que les dérivées supérieur sont définies jusqu'à l'ordre $n$. Il est particulièrement intéressant d'effectuer une optimisation des paramètres de la trajectoire pour minimiser un métrique qui correspond à l'effort des actionneurs, comme le snap pour un drone, de sorte à obtenir des mouvements gracieux. Par rapport à d'autre méthode d'optimisation de trajectoire, cette méthodologie est intéressante car l'espace de recherche est limité à des paramètres d'une trajectoire polynomiales qui garantissent la faisabilité, et l'optimisation est une problème de programmation quadratique qui est garantit de converger en temps fini. 

À venir!



Planifier une trajectoire fluide dans l'espace des sorties garantit la faisabilité dynamique. L'optimisation se réduit alors souvent à un problème de programmation quadratique (QP) sur les paramètres de courbes polynomiales, minimisant par exemple le \textit{snap} pour un quadrotor.

\newpage
\section{Méthodes de Transcription}

La transcription est le processus de conversion de la dynamique continue en contraintes algébriques pour un solveur NLP.

\subsection{Méthodes de Tir (\textit{Shooting})}

\begin{definition}{Transcription par Tir}{def:shooting}
    Ces méthodes imposent la dynamique via une simulation explicite (intégration numérique). Seules les commandes sont généralement des variables de décision.
\end{definition}

\begin{itemize}
    \item \textbf{Tir simple (\textit{Single Shooting}) :} On simule toute la trajectoire d'un coup. C'est simple mais numériquement instable car les erreurs d'intégration se propagent .
    \item \textbf{Tir multiple (\textit{Multiple Shooting}) :} La trajectoire est découpée en segments avec leurs propres états initiaux. Des contraintes de \textbf{défaut} assurent la continuité, rendant le problème plus robuste et parallélisable.
\end{itemize}



\subsection{Transcription Simultanée (\textit{Collocation})}

\begin{definition}{Collocation Directe}{def:collocation}
    L'état et la commande sont tous deux des variables de décision. La dynamique est imposée comme une contrainte d'égalité à des points spécifiques nommés points de collocation.
\end{definition}

\begin{itemize}
    \item \textbf{Transcription directe :} Utilise généralement la forme intégrale de la dynamique sur une grille fine.
    \item \textbf{Collocation directe :} Représente l'état par des polynômes par morceaux. On force la pente du polynôme à correspondre à la dynamique $\dot{x} = f(x,u)$ aux points de collocation.
    \item \textbf{Méthodes Pseudospectrales :} Utilisent des polynômes d'ordre élevé sur toute la trajectoire. La convergence est obtenue par l'ordre du polynôme (\textbf{méthode $p$}) ou le nombre de segments (\textbf{méthode $h$}).
\end{itemize}





