\chapter{Optimisation de trajectoires}



\section{Formulation du problème}



\section{Planéité différentielle}
\label{sec:searchalgo}

Le concept de planéité différentielle, en anglais \textit{differential flatness}, peut être très utile pour planifier des trajectoires pour des systèmes sous-actionnés. Le principe est le suivant: certains systèmes ont des variables sorties qui ont une propriété spéciale, i.e. nommée la planéité différentielle, qui est que tout les états et forces du système sont entièrement définis par ces variables et leurs dérivées supérieures. Par exemple, la position de l'effecteur pour un bras manipulateur, le centre de masse d'un drone quadcopter, le point milieu entre les deux roues arrières d'une voiture ont cette propriété. Pour une bras manipulateur pleinement actionné, les forces nécessaires aux joints sont reliés à l'accélération à l'effecteur. Pour un drone, les forces de propulsion sont reliés à la quatrième dérivée temporelle de la position du centre de masse, souvent appelé le \textit{snap}.

Ce concept peut être exploité pour générer plus facilement des trajectoires faisable qui respecte la dynamique du système. L'idée est de définir la trajectoire désirée du système en définissant une trajectoire continue juste qu'à un ordre $n$ (le minimum est variable selon le système), qui permet d'ensuite calculer tous les autres états et forces à partir de cet trajectoire et les dérivées supérieures de celle-ci. Typiquement, on peut utiliser une paramétrisation de type polynomiale d'ordre $n$ pour garantir que les dérivées supérieur sont définies jusqu'à l'ordre $n$. Il est particulièrement intéressant d'effectuer une optimisation des paramètres de la trajectoire pour minimiser un métrique qui correspond à l'effort des actionneurs, comme le snap pour un drone, de sorte à obtenir des mouvements gracieux. Par rapport à d'autre méthode d'optimisation de trajectoire, cette méthodologie est intéressante car l'espace de recherche est limité à des paramètres d'une trajectoire polynomiales qui garantissent la faisabilité, et l'optimisation est une problème de programmation quadratique qui est garantit de converger en temps fini. 

À venir!



\newpage
\section{Algorithmes d'optimisation de trajectoires}

\video{Introduction à la commande optimale}{https://youtu.be/3x6Vg-RRZ50}

\colab{Démo d'introduction à l'optimisation de trajectoires}{https://colab.research.google.com/drive/1yq2GHAkvO6fTF2W-tRbACDBa9_scec2k?usp=sharing}

À venir!

Mots clés: programmation quadratique, \textit{direct collocation}, \textit{shooting}

