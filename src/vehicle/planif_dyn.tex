\chapter{Optimisation de trajectoires}
\label{chap:trajopt}

Sources utille:\\
\url{https://arxiv.org/pdf/1707.00284}\\
\url{https://underactuated.mit.edu/trajopt.html}

Chapitre en construction!!

\colab{Démo d'introduction à l'optimisation de trajectoires}{https://colab.research.google.com/drive/1yq2GHAkvO6fTF2W-tRbACDBa9_scec2k?usp=sharing}

\section{Introduction et contexte}

L'optimisation de trajectoire est une branche des méthodes numériques pour résoudre des problèmes de \textbf{commande optimale}. Contrairement à la planification cinématique qui se limite souvent à l'espace des configurations $\col{q}(t)$, le cadre ici est générique et s'applique à la forme d'état générale $\dot{\col{x}} = \col{f}(\col{x}, \col{u})$, incluant les forces et couples d'actionnement $\col{u}$.

\section{Formalisation mathématique}

Le but est de trouver la trajectoire optimale $\{\col{x}^*(t), \col{u}^*(t)\}$ qui minimise un coût tout en satisfaisant des contraintes.

\begin{definition}{Problème d'optimisation de trajectoire}{def:traj_opt}
À venir!
\end{definition}



\section{Techniques de transcription}

La transcription convertit le problème continu en un programme non-linéaire (NLP) que des solveurs (comme SNOPT ou IPOPT) peuvent résoudre.

\subsection{Méthodes de Tir (\textit{Shooting})}
L'idée est d'utiliser une simulation pour imposer la dynamique.

À venir!

\subsection{Méthodes Simultanées (\textit{Collocation})}
L'état et la commande sont représentés par des variables de décision à chaque point de la grille.

À venir!


\section{Planéité différentielle (\textit{Differential Flatness})}

La planéité différentielle est un outil de planification puissant qui permet d'éviter l'intégration numérique de la dynamique.

\begin{definition}{Planéité différentielle}{def:flatness}
    Un système est différentiellement plat s'il existe un ensemble de sorties "plates" $\col{y}$ telles que tous les états $\col{x}$ et toutes les commandes $\col{u}$ peuvent être exprimés uniquement en fonction de $\col{y}$ et de ses dérivées temporelles.
\end{definition}

Pour un système plat (ex: drone quadrotor), on peut planifier une trajectoire fluide (spline) dans l'espace cartésien (Chapitre 6). La "planéité" garantit qu'il existe une commande $\col{u}(t)$ réalisable pour suivre exactement cette courbe sans avoir à résoudre un NLP complexe.


