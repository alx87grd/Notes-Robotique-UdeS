%%%%%%%%%%%%%%%%%%%%%%%%%%%%%%%%%%%%%%%%%%%%%%%%%%%%%%%%%%%%%%%%%%%%%%%%%%%%%%%%
\newpage
%%%%%%%%%%%%%%%%%%%%%%%%%%%%%%%%%%%%%%%%%%%%%%%%%%%%%%%%%%%%%%%%%%%%%%%%%%%%%%%%
\section{Problèmes déterministes discrets}
Lorsqu'on travail avec un problème où les états et actions possibles sont discrets, c'est à dire qu'il peuvent prendre un nombre fini de valeurs. Par exemple, pour une automobile la transmission a généralement autour de 6 options discrètes (marche arrière, neutre, 1ère vitesse, 2e vitesse, etc.) tandis que la commande d'accélération est une variable continue qui peut prendre n'importe quel valeur à l'intérieur d'une certaine plage. Pour les systèmes ou tout est discret, il est possible de représenter les états comme des noeuds sur un graphe et les actions possible comme des arcs qui nous mène vers un autre état. Dans ce contexte on peut simplifier la notation.

"A

%%%%%%%%%%%%%%%%
\begin{align}
J^*_k(i) &= 
\operatornamewithlimits{min}\limits_{j \in U(i)}
%\mathbb{E}
\left[
a_{ij}^k + J^*_{k+1}(j)
\right] \\
j^*_k(i) &= 
\operatornamewithlimits{argmin}\limits_{j \in U(i)}
%\mathbb{E}
\left[
a_{ij}^k + J^*_{k+1}(j)
\right] 
\label{eq:exactdpgraph}
\end{align} 
%%%%%%%%%%%%%%%%%