\chapter{Commande des robots manipulateurs II: dynamique}

Chapitre en construction!!

\section{Introduction à la commande des manipulateurs}

\video{Introduction à la commande des robots}{https://youtu.be/eL6i319X_w4}

\video{L'espace des phases}{https://youtu.be/eL6i319X_w4}

\video{Tour d'horizon des méthodes de commande dans l'espace des phases}{https://youtu.be/eL6i319X_w4}

\colab{Testez votre loi de commande pour un manipulateur}{https://colab.research.google.com/drive/1bnJ9v5kHRFFhnNOZx-_Kcht2l5sTOHxr?usp=sharing}

\video{Grandes familles de méthodes de commande pour gérer l'incertitude}{https://youtu.be/hbZBF-OEZEw}


\section{Commande dé-localisée}

\colab{Simulation d'un manipulateur avec des PIDs }{https://colab.research.google.com/drive/1qaCNY2ohQIbC6dV2ZW4KY6FDM9NhJzH2?usp=sharing}



\section{Commande avec la méthode du couple calculé}

La méthode du couple calculé consiste à utiliser les équations d'un modèle dynamique d'un système robotique pour déterminer les forces à appliquer pour obtenir une accélération cible. Il est ensuite possible de calculer cette accélération cible de sorte à converger vers la position ou trajectoire désirée.

\video{Méthode du couple calculé}{https://youtu.be/QuEhwAUxx5Y}

\subsection{Commande de l'accélération des joints}

À venir!


\subsection{Commande en impédance incluant les effets inertiels }

TODO voir notes manuscriptes alex 3 juillet 2022

\section{Commande robuste}

\subsection{Méthode du mode glissant}

\video{Commande avec le mode glissant}{https://youtu.be/0Asg81SBjmk}

\section{Commande adaptative}

\video{Exemple de loi de commande adaptative}{https://youtu.be/vmYjad6SOmU}


\section{Commande hybride en position et force}

\section{Commande optimale}


\video{Exemple de loi de commande optimale pour un double intégrateur}{https://youtu.be/wKjEAXFvXlQ}
\video{Exemple de loi de commande optimale pour un pendule}{https://youtu.be/iUlkKdEK_dU}

\colab{Démo d'introduction aux méthodes de commande optimales}{https://colab.research.google.com/drive/1wXmlIqNGC2LrJkmyj56Y109b5ZDHVboq?usp=sharing}


\section{Analyse de stabilité}

\video{Introduction à l'analyse de stabilité pour les systèmes non-linéaires}{https://youtu.be/q0Oqa5J3zEk}