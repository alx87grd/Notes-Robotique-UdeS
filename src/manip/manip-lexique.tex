\newpage

\section{Lexique}

\begin{center}
\begin{tabular}{  p{3.5cm} p{3.5cm} p{7cm} }
%\caption{Définitions techniques }
%\label{def}
\hline 
\textbf{Terme technique} & \textbf{En anglais} & \textbf{Définition} \\ \hline\hline \\
%%
Robot Manipulateur &  Manipulator Robot &
Robot avec une base fixe qui a comme tâche de positionner dans l'espace un objet ou un outil.
\\  &  \\ 
Actionneur &  Actuator &
Dispositif qui transforme l'énergie en travail mécanique. Typiquement des moteurs électriques et des vérins pneumatiques ou hydrauliques pour les robots.
\\  &  \\ 
Effecteur  & End-effector &
L'endroit où est situé l'objet ou l'outil manipulé par un robot.
\\   &  \\
Corps rigide & Rigid body &
Un modèle idéal d'un objet qui ne se déforme pas même lorsque soumis à des forces externes.
\\   &  \\ 
Joint & Joint &
Jonction mécanique permettant un mouvement relatif entre deux pièces. 
\\  &  \\ 
Système de coordonnées &  Coordinate system &
Ensemble de scalaires (angles ou positions) qui permettent de paramétrer la position/configuration d'un système. 
\\  &  \\ 
Degrés-de-Liberté (DDL) & Degree-of-Freedom (DoF) &
Le nombre de variable indépendante qui permettent de paramétrer dans l'espace la position/configuration d'un système. 
\\  &  \\ 
Base vectorielle  & Vector basis &
Trois vecteurs unitaires qui définissent une orientation. 
\\  &  \\ 
Base vectorielle orthonormée & Orthogonal Vector basis &
Trois vecteurs unitaires orthogonaux qui définissent une orientation. 
\\  &  \\ 
Origine &  Origin &
Un point de référence pour la mesure des positions.
\\  &  \\ 
Repère &  Frame &
Combinaison d'une origine et d'une base vectorielle orthonormée.
\\  &  \\ 
Système de coordonnées cartésiennes & Cartesian coordinate system &
Ensemble de trois scalaires qui déterminent la position d'un point par rapport à une origine et trois axes orthogonaux. %(définis par une base vectorielle orthonormée)
\\   &  \\  
Référentiel & Reference frame &
Point de vue choisi pour analyser/observer un mouvement.
\\  &  \\ 
\hline
\label{tab}
\end{tabular}
\end{center}

\section{Symboles et notations}

\begin{center}
\begin{tabular}{p{5cm}  p{9cm}}
%\caption{Nomenclature}
%\label{nom}
\hline
\textbf{Symbole} & \textbf{Définition} \\ \hline\hline \\
%%%%%%%%%%%%%%%%%%%%%%%%%%%%%%%%%%%%%%%%%%%%%%%%%%%%%%%%%%%%%%%%%%%%%%%%%%%%%%%%%%%%%%%%%%%%%%%%%%%%
\multicolumn{2}{l}{Vecteurs géométriques} \\ \hline \\
%%%%%%%%%%%%%%%%%%%%%%%%%%%%%%%%%%%%%%%%%%%%%%%%%%%%%%%%%%%%%%%%%%%%%%%%%%%%%%%%%%%%%%%%%%%%%%%%%%%%
$\vec{v}$            & Vecteur      \\   &  \\ 
%%%%%%%%%%%%%%%%%%%%%%%%%%%%%%%%%%%%%%%%%%%%%%%%%%%%%%%%%%%%%%%%%%%%%%%%%%%%%%%%%%%%%%%%%%%%%%%%%%%%
$\hat{a}$            & Vecteur unitaire   \\   &  \\ 
%%%%%%%%%%%%%%%%%%%%%%%%%%%%%%%%%%%%%%%%%%%%%%%%%%%%%%%%%%%%%%%%%%%%%%%%%%%%%%%%%%%%%%%%%%%%%%%%%%%%
%$\vec{r}$            & Vecteur position     \\  &  \\ 
%%%%%%%%%%%%%%%%%%%%%%%%%%%%%%%%%%%%%%%%%%%%%%%%%%%%%%%%%%%%%%%%%%%%%%%%%%%%%%%%%%%%%%%%%%%%%%%%%%%%
%$\vec{a}$            & Vecteur accélération \\   &  \\ 
%%%%%%%%%%%%%%%%%%%%%%%%%%%%%%%%%%%%%%%%%%%%%%%%%%%%%%%%%%%%%%%%%%%%%%%%%%%%%%%%%%%%%%%%%%%%%%%%%%%%
%$\dot{\vec{r}}$      & Dérivée du vecteur $\vec{r}$ par rapport au temps \\   &  \\  
%%%%%%%%%%%%%%%%%%%%%%%%%%%%%%%%%%%%%%%%%%%%%%%%%%%%%%%%%%%%%%%%%%%%%%%%%%%%%%%%%%%%%%%%%%%%%%%%%%%%
%$^{i}\dot{\vec{r}}$  & Dérivée partielle du vecteur $\vec{r}$ par rapport au temps vue dans le référentiel $i$ \\   &  \\ 
%%%%%%%%%%%%%%%%%%%%%%%%%%%%%%%%%%%%%%%%%%%%%%%%%%%%%%%%%%%%%%%%%%%%%%%%%%%%%%%%%%%%%%%%%%%%%%%%%%%%
$\vec{r}_{A/B}$      & Vecteur position du point $A$ par rapport au point $B$ \\   &  \\ 
%%%%%%%%%%%%%%%%%%%%%%%%%%%%%%%%%%%%%%%%%%%%%%%%%%%%%%%%%%%%%%%%%%%%%%%%%%%%%%%%%%%%%%%%%%%%%%%%%%%%
\multicolumn{2}{l}{Vecteur-colonnes et composantes scalaires} \\ \hline \\
%%%%%%%%%%%%%%%%%%%%%%%%%%%%%%%%%%%%%%%%%%%%%%%%%%%%%%%%%%%%%%%%%%%%%%%%%%%%%%%%%%%%%%%%%%%%%%%%%%%%
$\col{c} = \left[ \begin{array}{c}
	c_1 \\ c_2 \\ c_3
\end{array}  \right] = \left[\begin{array}{ccc} c_1 & c_2 & c_3 \end{array}\right]^T $    
 & Vecteur-colonne \\   &  \\
%%%%%%%%%%%%%%%%%%%%%%%%%%%%%%%%%%%%%%%%%%%%%%%%%%%%%%%%%%%%%%%%%%%%%%%%%%%%%%%%%%%%%%%%%%%%%%%%%%%%
%$\col{v}^{T} = \left[v_1,v_2,v_3\right]$
%& Vecteur-rangé  \\   &  \\
%%%%%%%%%%%%%%%%%%%%%%%%%%%%%%%%%%%%%%%%%%%%%%%%%%%%%%%%%%%%%%%%%%%%%%%%%%%%%%%%%%%%%%%%%%%%%%%%%%%%
$\col{v}^{a}  = \left[ \begin{array}{c}
	v_1^a \\ v_2^a \\ v_3^a
\end{array}  \right]$   & Vecteur-colonne des composantes du vecteur $\vec{v}$ dans la base vectorielle $a$  \\   &  \\
%%%%%%%%%%%%%%%%%%%%%%%%%%%%%%%%%%%%%%%%%%%%%%%%%%%%%%%%%%%%%%%%%%%%%%%%%%%%%%%%%%%%%%%%%%%%%%%%%%%%
$v^{a}_i$        & Composante du vecteur $\vec{v}$ selon le vecteur unitaire $i$ de la base vectorielle $a$ \\   &  \\
%%%%%%%%%%%%%%%%%%%%%%%%%%%%%%%%%%%%%%%%%%%%%%%%%%%%%%%%%%%%%%%%%%%%%%%%%%%%%%%%%%%%%%%%%%%%%%%%%%%%
%%%%%%%%%%%%%%%%%%%%%%%%%%%%%%%%%%%%%%%%%%%%%%%%%%%%%%%%%%%%%%%%%%%%%%%%%%%%%%%%%%%%%%%%%%%%%%%%%%%%
$\col{q} = \left[ \begin{array}{c}
	q_1 \\  \vdots \\ q_n
\end{array}  \right] $            & Vecteur-colonne qui regroupe $n$ scalaires représentant la configuration d'un robot à $n$ DDL dans l'espace des joints. \\   &  \\
%%%%%%%%%%%%%%%%%%%%%%%%%%%%%%%%%%%%%%%%%%%%%%%%%%%%%%%%%%%%%%%%%%%%%%%%%%%%%%%%%%%%%%%%%%%%%%%%%%%%
\multicolumn{2}{l}{Bases et repères} \\ \hline \\
%%%%%%%%%%%%%%%%%%%%%%%%%%%%%%%%%%%%%%%%%%%%%%%%%%%%%%%%%%%%%%%%%%%%%%%%%%%%%%%%%%%%%%%%%%%%%%%%%%%%
$\left\{ \hat{a}_1 , \hat{a}_2 , \hat{a}_3 \right\}$  & Base vectorielle $a$ définie par un ensemble de trois vecteur unitaires orthogonaux \\   &  \\ 
%%%%%%%%%%%%%%%%%%%%%%%%%%%%%%%%%%%%%%%%%%%%%%%%%%%%%%%%%%%%%%%%%%%%%%%%%%%%%%%%%%%%%%%%%%%%%%%%%%%%
$\left\{ A_O , \hat{a}_1 , \hat{a}_2 , \hat{a}_3 \right\}$  & Repère $A$ défini par un point d'origine $A_O$ et la base vectorielle $a$ \\   &  \\
%%%%%%%%%%%%%%%%%%%%%%%%%%%%%%%%%%%%%%%%%%%%%%%%%%%%%%%%%%%%%%%%%%%%%%%%%%%%%%%%%%%%%%%%%%%%%%%%%%%%
%%%%%%%%%%%%%%%%%%%%%%%%%%%%%%%%%%%%%%%%%%%%%%%%%%%%%%%%%%%%%%%%%%%%%%%%%%%%%%%%%%%%%%%%%%%%%%%%%%%%
\multicolumn{2}{l}{Matrices} \\ \hline \\
%%%%%%%%%%%%%%%%%%%%%%%%%%%%%%%%%%%%%%%%%%%%%%%%%%%%%%%%%%%%%%%%%%%%%%%%%%%%%%%%%%%%%%%%%%%%%%%%%%%%
$M=\left[ \begin{array}{c c c}
	M_{1,1} & \hdots & M_{1,n}  \\ \vdots & \ddots & \vdots \\ M_{m,1}  & \hdots & M_{m,n}
\end{array}  \right] $   
& Matrice de $m$ rangées et $n$ colonnes ($m \times n$) \\   &  \\
%%%%%%%%%%%%%%%%%%%%%%%%%%%%%%%%%%%%%%%%%%%%%%%%%%%%%%%%%%%%%%%%%%%%%%%%%%%%%%%%%%%%%%%%%%%%%%%%%%%%
${}^aR^b = \left[ \begin{array}{c c c}
	\col{b}_1^a  & \col{b}_2^a & \col{b}_3^a
\end{array}  \right]$            & Matrice de rotation (3$\times$3) qui représente l'orientation de la base vectorielle $b$ par rapport à la base vectorielle $a$. %Un changement de base est effectué ainsi $ \col{v}^b={}^{b}R^a \col{v}^a$  
 \\   &  \\
%%%%%%%%%%%%%%%%%%%%%%%%%%%%%%%%%%%%%%%%%%%%%%%%%%%%%%%%%%%%%%%%%%%%%%%%%%%%%%%%%%%%%%%%%%%%%%%%%%%%
$^{A}T^B = \left[ \begin{array}{c c}
	{}^aR^b  & \col{r}^a_{B_o/A_o} \\ 0 \; 0 \; 0 & 1
\end{array}  \right] $           & Matrice de transformation homogène (4$\times$4) du repère $B$ vers le repère $A$\\   &  \\
%%%%%%%%%%%%%%%%%%%%%%%%%%%%%%%%%%%%%%%%%%%%%%%%%%%%%%%%%%%%%%%%%%%%%%%%%%%%%%%%%%%%%%%%%%%%%%%%%%%%
%$\underline{v}^{\times}= \left[ \begin{array}{c c c}
%	0 & -v_3 & v_2  \\ v_3 & 0 & -v_1 \\ -v_2 & v_1 & 0
%\end{array}  \right] $   & Matrice antisymétrique correspondant aux composantes du vecteur $\vec{v}$ pour calculer un %produit vectoriel : $\vec{v} \times \vec{w} \; \leftrightarrow \; \col{v}^{\times}\col{w}$ \\   &  \\
%%%%%%%%%%%%%%%%%%%%%%%%%%%%%%%%%%%%%%%%%%%%%%%%%%%%%%%%%%%%%%%%%%%%%%%%%%%%%%%%%%%%%%%%%%%%%%%%%%%%
\hline
\end{tabular}
\end{center}