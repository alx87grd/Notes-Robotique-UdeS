\chapter{Passivité}

\section{Définition}
Un système avec une entrée $u(t)$ et une sortie $y(t)$ est passif s'il existe une fonction de stockage $S(x(t)) \geq 0$ telle que pour tout $t \geq t_0$ :
$$ S(x(t)) - S(x(t_0)) \leq \int_{t_0}^{t} y(\tau)^T u(\tau) d\tau $$
où $x(t)$ est le vecteur d'état du système.

\section{Puissance et Énergie}
La puissance instantanée entrant dans le système est donnée par :
$$ P(t) = y(t)^T u(t) $$
L'énergie fournie au système sur un intervalle de temps $[t_0, t]$ est :
$$ E(t_0, t) = \int_{t_0}^{t} y(\tau)^T u(\tau) d\tau $$

\section{Fonction de Stockage}
La fonction de stockage $S(x(t))$ représente l'énergie stockée à l'intérieur du système. Elle doit être non négative :
$$ S(x(t)) \geq 0, \quad S(0) = 0 $$

\section{Inégalité de Passivité}
L'inégalité de passivité stipule que le changement d'énergie stockée est inférieur ou égal à l'énergie fournie au système :
$$ \dot{S}(x(t)) \leq y(t)^T u(t) $$

\section{Passivité Stricte}
Un système est strictement passif s'il existe une fonction de stockage $S(x(t)) \geq 0$ et une fonction définie positive $\psi(x(t)) > 0$ telle que :
$$ \dot{S}(x(t)) \leq y(t)^T u(t) - \psi(x(t)) $$
Ceci implique qu'une certaine quantité d'énergie est toujours dissipée à l'intérieur du système.

\section{Passivité Stricte en Entrée}
Un système est strictement passif en entrée (ISP) s'il existe une fonction de stockage $S(x(t)) \geq 0$ et une constante $\delta > 0$ telle que :
$$ \dot{S}(x(t)) \leq y(t)^T u(t) - \delta \|u(t)\|^2 $$
Ceci signifie qu'une partie de l'énergie d'entrée est toujours dissipée.

\section{Passivité Stricte en Sortie}
Un système est strictement passif en sortie (OSP) s'il existe une fonction de stockage $S(x(t)) \geq 0$ et une constante $\varepsilon > 0$ telle que :
$$ \dot{S}(x(t)) \leq y(t)^T u(t) - \varepsilon \|y(t)\|^2 $$
Ceci signifie qu'une partie de l'énergie liée à la sortie est toujours dissipée.

\section{Système Sans Perte}
Un système est sans perte s'il existe une fonction de stockage $S(x(t)) \geq 0$ telle que :
$$ \dot{S}(x(t)) = y(t)^T u(t) $$
Dans ce cas, toute l'énergie fournie au système est stockée, et aucune n'est dissipée.

\section{Condition de Domaine Fréquentiel pour les Systèmes Linéaires Invariants dans le Temps (LTI)}
Pour un système LTI stable avec une fonction de transfert $H(s)$, le système est passif si et seulement si :
$$ H(j\omega) + H(j\omega)^H \geq 0 \quad \forall \omega \in \mathbb{R} $$
où $H(j\omega)^H$ est la transposée hermitienne de $H(j\omega)$. Pour les systèmes scalaires, ceci se simplifie à :
$$ \text{Re}\{H(j\omega)\} \geq 0 \quad \forall \omega \in \mathbb{R} $$
Ceci signifie que l'angle de phase $\phi(\omega)$ de $H(j\omega) = |H(j\omega)| e^{j\phi(\omega)}$ satisfait :
$$ |\phi(\omega)| \leq \frac{\pi}{2} $$


\section{Exemple}

Un système mass-ressort-amortisseur, avec comme entrée la force et comme sortie la vitesse:

\begin{align}
    H(s) = \frac{s}{ms^2 + bs + k}
\end{align}
