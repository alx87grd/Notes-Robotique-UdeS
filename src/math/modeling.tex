% %%%%%%%%%%%%%%%%%%%%%%%%%%%%%%%%%%%%%%%%%%
\chapter{Méthodes de modélisation}
\label{sec:modeling}



\section{Types de modèles mathématiques}

\video{Types d'équations pour modéliser un système dynamique}{https://youtu.be/D_HLuoPrD4w}

\subsection{Équations différentielles}

\video{Grandes familles d'équation différentielles}{https://youtu.be/2ifsprEkJYU}

\subsection{Équations différentielles partielles}

\subsection{Équations de différences}

\subsection{Chaînes de Markov}


%%%%%%%%%%%%%%%%%%%%%%%%%%%%%%%%%%%%%%%%%%%%%%%%%%%%%%%%
\subsection{Représentation des systèmes hybrides}
%%%%%%%%%%%%%%%%%%%%%%%%%%%%%%%%%%%%%%%%%%%%%%%%%%%%%%%

Hybrid dynamical system can be represented in the general form:
%
\begin{align}
\text{Continuous evolution: } \left(  \dot{\col{x}} , \dot{k} \right) &=  \left( \, f_k( \col{x} , \col{u} , \col{d} ) \, , \, 0 \, \right) \\
\text{Discrete jumps: } \left(  \col{x}^+ , k^+ \right) &=  \left( h_{ij}( \col{x}^- , \col{u}^- ) , j \right) \quad\text{if}\quad \left(  \col{x} , k , \col{u} \right) \in D_{ij}  
\end{align}
%
where $\col{x}$ is a continuous state vector, and $k$ is a discrete mode and $D_{ij}$ is the domain mapping conditions leading to a transition $k:i \rightarrow j$. For robotic systems, the discrete mode can represent discrete configurations of the robot , like discrete modes in the control law or contact/non-contact conditions. The jump map then represents the impulsive response when contact is made. 

\subsection{Switched system}

A restricted class of hybrid system, called switched system, are hybrid systems for which the jump map for continuous state is the identify function:
%
\begin{align}
\text{Continuous evolution: } \left(  \dot{\col{x}} , \dot{k} \right) &=  \left( \, f_k( \col{x} , \col{u} , \col{d} ) \, , \, 0 \, \right) \\
\text{Discrete jumps: } \left(  \col{x}^+ , k^+ \right) &=  \left( \col{x}^- , j \right) \quad\text{if}\quad \left(  \col{x} , k , \col{u} \right) \in D_{ij} 
\end{align}
%

\subsubsection{Switched system where the discrete mode is a control input}

In the situation where the discrete operating mode $k$ is a control input, then there is no need to keep track of discrete mode evolution and only the piece-wise continuous differential equations are sufficient to model the system evolution:
%
\begin{align}
\dot{\col{x}} = f_k( \col{x} , \col{u} , \col{d} ) 
\end{align}
%
A robot with a continuous dynamics but with a control law using discrete modes of operation would be of this category. 



%\section{Types de modèles mathématiques}