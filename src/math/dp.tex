% %%%%%%%%%%%%%%%%%%%%%%%%%%%%%%%%%%%%%%%%%%
\chapter{Programmation dynamique: La science des prises de décisions optimales}
\label{sec:decision}
% {\hypersetup{linkcolor=black}
% \parttoc
% }

% \iftoggle{EN}{%

% \section*{
% Foreword
% }
% These notes aims to provide the theoretical background to unify various decision making schemes under the dynamic programming framework. %It is aim to make links with approches from the control engineering field and those from the field of operationnel research and reinforcement learning.


% Video lectures (in french) are available here:
% }{%

Ce chapitre présente les diverses approches pour prendre des décisions intelligentes sous un cadre théorique unifié basé sur le principe de la programmation dynamique. Elle vise d'abord a établir les liens entre les approches issues du domaine de l'ingénierie (la science des asservissements et la \textbf{commande optimale}) et les approches issues des sciences informatiques (recherche opérationnelle et l'\textbf{apprentissage par renforcement}) qui ont en fait les \textbf{même bases mathématiques}. Ces notes visent principalement à donner à un lecteur issue du domaine de l'ingénierie les bases pour comprendre et utiliser les approches numériques issues des sciences informatiques. 

\video{Capsules vidéos sur la programmation dynamique}{https://youtube.com/playlist?list=PL6adNeJ0A8UtNslNQfAHAzcjHcQixBSnu}



\section{Introduction}
% \iftoggle{EN}{%
% Dynamic programming (DP) is a mathematical principle to optimize decision that are taken in sequence after observing the state of a system. DP can be used for analyzing classical feedback control problems, such as controling the position of an electric motor by choosing the input voltage, but also for analyzing decisions in a stochastic context, for instance buying or not a stock, or even for progamming the AI of a computer game that must select the move in a chess game after observing the board state.
% }{%
La programmation dynamique est un principe mathématique pour optimiser des décisions qui sont prises en séquence après avoir observé l'état d'un système. Le principe peut être utilisé autant pour analyser un système asservis classique, comme contrôler la position d'un moteur en choisissant la tension appliquée à ses bornes, que pour des problèmes probabiliste dans un contexte de finance, comme choisir quand acheter ou vendre une action en observant l'évolution de son prix, ou bien un problème d'IA comme choisir la pièce à déplacer lors d'une partie d'échec en observant la position des pièces sur l'échiquier.
% }

\subsection{Formulation du problème}

\paragraph{Un comportement défini par une loi de commande à concevoir}
On s'intéresse au problème de concevoir une \textbf{une loi de commande}, qui consiste en une fonction $c$ qui détermine l'action $u$ à prendre en fonction de l'état $x$ d'un système observé, et possiblement du temps $t$:
%%%%%%%%%%%%%%%%%%%%%%%%%%
\begin{align}
u = c( x, t )
\end{align}
%%%%%%%%%%%%%%%%%%%%%%%%%%

Cette loi de commande va être implémenté dans un agent qui observe l'état $x$ d'un système et agit en conséquence sur ce système pour l'influencer, formant ainsi une boucle comme illustré à la figure \ref{fig:agent}. 

%%%%%%%%%%%%%%%%%%%%%%%%%%%%%%%%%%%%
\begin{figure}[H]
	\centering
	\vspace{-10pt}
		%\includegraphics[width=0.60\textwidth]{XXX.jpg}
		\caption{La tâche est de concevoir une loi de commande qui définie le comportement d'un agent}
		\label{fig:agent}
	%\vspace{-10pt}
\end{figure}
%%%%%%%%%%%%%%%%%%%%%%%%%%%%%%%%%%%%

\paragraph{Un objectif formulé comme la minimisation d'un coût}
Ensuite, notre loi de commande sera conçue de sorte à atteindre un objectif qui sera exprimé mathématiquement comme la minimisation d'un coût additif qui dépend de la trajectoire du système et les actions utilisée:
%%%%%%%%%%%%%%%%%%%%%%%%%%
\begin{align}
    J=\int_0^{t_f}g(x,u,t) dt + h(x_f,t_f)
\end{align}
%%%%%%%%%%%%%%%%%%%%%%%%%%%
où $g$ est un coût cumulatif sur la trajectoire du système, $h$ est un coût terminal et $t_f$ est un horizon de temps. La forme cumulative de la fonction coût est centrale pour utiliser le principe de la programmation dynamique, mais ce n'est pas vraiment restrictif car pratiquement tout les objectifs peuvent être formulés comme la minimisation d'une fonction de coût avec cette forme. Lorsque que notre agent prend les meilleurs décisions possible en fonction de l'objectif on dira que sa loi de commande est optimale au sens qu'elle minimise la fonction de coût. 

%%%%%%%%%%%%%%%%%%%%%%%%%%%%%%%%%%%%
\begin{figure}[H]
	\centering
	\vspace{-10pt}
		%\includegraphics[width=0.60\textwidth]{XXX.jpg}
		\caption{Fonction de coût}
		\label{fig:cost}
	%\vspace{-10pt}
\end{figure}
%%%%%%%%%%%%%%%%%%%%%%%%%%%%%%%%%%%%

\video{Introduction et formulation du problème}{https://youtu.be/1ThWOUnkVyY}

\note{Minimiser un coût ou maximiser une récompense?}{On peut alternativement formuler l'objectif comme le problème de maximiser une récompense. Les deux approches sont équivalentes et interchangeables. Typiquement le domaine de la commande optimale utilise la formulation de minimiser un coût, qui est souvent relié à l'erreur par rapport à une trajectoire cible. Alternativement, le domaine de l'apprentissage par renforcement préfère optimiser une récompense, qui est souvent par exemple le pointage dans un jeux pour lequel une IA est développé.}



\paragraph{Exemple: loi de commande pour un robot}

Un exemple d'asservissement classique serait un bras robotique où l'action $u$ déterminée par la loi de commande correspond à un vecteur de couples à appliquer dans les moteurs électriques. Cette action sera calculée en fonction de l'état actuel du robot, donc ici un vecteur de positions et vitesses de ses diverses articulations. L'objectif serait formulé comme la minimisation de l'erreur de position du robot par rapport à une position cible et potentiellement d'une pénalité pour utiliser beaucoup d'énergie. Typiquement notre solution de loi de commande serait ici une équation analytique.

\paragraph{Exemple: navigation d'un véhicule}

Un exemple de prise de décision à plus haut niveau serait de choisir un trajet sur une carte. La loi de commande déterminerait ici quelle direction prendre en fonction de la position actuelle sur la carte. L'objectif d'atteindre la destination le plus rapidement possible pourrait être formuler comme la minimisation du temps écoulé avant d'atteindre celle-ci. La loi de commande (qui serait une solution globale) pourrait être sous la forme d'une table de correspondance (\textit{look-up table)} où est en mémoire la direction optimale à prendre pour chaque intersection sur laquelle on peut se trouver sur la carte.

\paragraph{Exemple: achats d'une action}

Un exemple dans un tout autre contexte serait pour un algorithme d'investissement. L'action de la loi de commande serait ici d'acheter ou non une action en fonction d'une observation de son prix. L'objectif pourrait ici être formuler comme la maximisation des gains financiers. La loi de commande serait ici un seuil de prix, qui pourrait varier en fonction du temps, en dessous duquel l'agent décide d'acheter l'action.


\newpage
\subsection{Formulation du problème en temps discret}


La plupart des outils pour travailler avec ce genre de problèmes sont mieux adapté à une approche de type temps discret. Ces notes vont donc présenter les principes et les algorithmes d'abord avec une approche à temps discret ou un index $k$ indique l'étape actuelle. Il est possible de convertir une problème à temps continu en problème à temps discret en introduisant un pas de temps $dt$ et en considérant que les décisions sont prises en séquence à chaque période de temps $dt$. Alternativement, une approche pour travailler directement en temps continue est présentée à la section \ref{sec:dp_cont}. De plus, pour plusieurs types de problèmes la nature de l'évolution du système est discrètes, par exemple une partie d'échec. La formulation des problèmes en temps discret est donc très générale et s'applique a un grand nombre de problèmes.

Le problème équivalent à résoudre en temps discret est de déterminer les lois de commande $c_k$, qui dictent l'action $u$ à prendre lorsque l'état du système est de $x$ à l'étape $k$:
\begin{align}
u_k = c_k( x_k )
\end{align}
de sorte à minimiser un coût additif de la forme:
\begin{align}
    J = \sum_{k=0}^{N-1} g_k(x_k, u_k) + g_N( x_N )
\end{align}
où $N$ est l'horizon qui représente ici un nombre d'étape. De plus ici l'évolution du système est représentée par une équation de différence:
\begin{align}
    x_{k+1} = f_k( x_k, u_k)
\end{align}

%%%%%%%%%%%%%%%%%%%%%%%%%%%%%%%%%%%%
\begin{figure}[H]
	\centering
	\vspace{-10pt}
		%\includegraphics[width=0.30\textwidth]{XXX.jpg}
		\caption{Evolution}
		\label{fig:evolution}
	%\vspace{-10pt}
\end{figure}
%%%%%%%%%%%%%%%%%%%%%%%%%%%%%%%%%%%%

Si on reforme tout le problème en une seule équation mathématique:
%%%%%%%%%%%%%%%%%%%%%
\begin{align}
    J^*(x_0) = \min_{c_0, ... c_k, ... c_{N-1}} 
    \left[ \sum_{k=0}^{N-1} g_k(x_k, u_k) + g_N( x_N )
    \right] \quad
    \text{avec} \quad
    x_{k+1} = f_k( x_k, c_k(x_k) )
\end{align}
%%%%%%%%%%%%%%%%%%%%%
on cherche les fonctions $c_k$, i.e. les loi de commandes, qui vont minimiser le coût cumulatif  sur la trajectoire du système, avec l'évolution qui est définit par une dynamique $f_k$ et les lois de commandes $c_k$.


\newpage
\subsection{Terminologie}

\begin{center}
\begin{tabular}{  p{3.0cm} p{6.0cm} p{7.0cm} }
%\caption{Définitions techniques }
%\label{def}
\hline 
\textbf{Variable} & \textbf{Termes} & \textbf{Définition} \\ \hline\hline \\
%%
$f_k(x_k,u_k)$ & Dynamique, Système, Environnement, Processus, \textit{Plant}, &
Équations qui définit l'évolution du système, i.e. de l'environnement de l'agent.
\\  &  \\ 
$u=c_k(x)$ & Loi de commande, contrôleur, politique de l'agent, \textit{policy} &
Fonction qui définit la décision de l'agent comme une fonction de l'observation de l'état.
\\  &  \\ 
$x$ & État, \textit{State} &
Variable qui représente toute l'information pour prédire l'évolution future du système.
\\  &  \\ 
$u$ & Action, décision, entrée du système, \textit{control input} &
Variable qui représente la décision de l'agent.
\\  &  \\ 
$k$ & index &
Entier correspondant à l'étape actuelle (i.e. souvent représentant le temps discrétisé. 
\\  &  \\ 
$U_k(x)$ & Contraintes, \textit{control set} &
Ensemble représentant les actions qui sont possible lorsque l'état du système est $x$
\\  &  \\ 
$J(x_0, ... , x_N)$ & Fonction de coût cumulative, (opposé de la) fonction de récompense, \textit{value fonction} &
Fonction scalaire qui représente à quel point une trajectoire est bonne en fonction de l'objectif.
\\  &  \\ 
$J_c(x)$ & Coût à venir &
Fonction scalaire qui représente la prédiction de coût cummulatif d'une trajectoire qui débute à $x$ en utilisant la loi $c$
\\  &  \\ 
$J^*(x)$ & Coût à venir optimal &
Fonction scalaire qui représente la prédiction de coût cummulatif d'une trajectoire qui débute à $x$ si toutes les actions dans le futur sont optimales
\\  &  \\
$g_k(x_k,u_k)$ & Coût instantanée & Fonction scalaire qui définit le coût instantané à l'étape $k$
\\  &  \\ 
$g_N(x_N)$ & Coût terminal &
Fonction scalaire qui définit le coût final en fonction de l'état terminal.
\\  &  \\ 
\hline
\label{tab}
\end{tabular}
\end{center}




%%%%%%%%%%%%%%%%%%%%%%%%%%%%%%%%%%%%%%%%%%%%%%%%
\subsection{Principe d'optimalité}

Si une trajectoire est optimale de $x_0$ à $x_N$ en passant par $x_i$. Alors nécessairement la séquence de fin de cette trajectoire correspond aussi à la solution optimale d'une trajectoire qui débuterait à $x_i$ pour aller à $x_N$.
%%%%%%%%%%%%%%%%
\begin{align}
\left[ x_0 , x_1 , ... , x_i , ... , x_N \right] \\
\left[ u_0 , u_1 , ... , u_i , ... , u_N \right] \\
\left[  x_i , ... , x_N \right] \\
\left[  u_i , ... , u_N \right]
\end{align}
%%%%%%%%%%%%%%%%

\video{Le principe d'optimalité}{https://youtu.be/EMkpkyMTg4U}


%%%%%%%%%%%%%%%%%%%%%%%%%%%%%%%%%%%%%%%%%%%%%%%%
\subsection{Programmation dynamique excate}




%%%%%%%%%%%%%%%%
\begin{align}
J^*_N(x_N) &= g_N(x_N) \\
J^*_k(x_k) &= 
\operatornamewithlimits{min}\limits_{u_k \in U_k(x_k)}
%\mathbb{E}
\left[
g_k(x_k , u_k ) + J^*_{k+1}( 
\underbrace{
f_k(x_k , u_k ) 
}_{x_{k+1}}
)
\right] \\
c^*_k(x_k) &= 
\operatornamewithlimits{argmin}\limits_{u_k\in U_k(x_k)}
%\mathbb{E}
\left[
g_k(x_k , u_k ) + J^*_{k+1}( 
\underbrace{
f_k(x_k , u_k ) 
}_{x_{k+1}}
)
\right] 
\label{eq:exactdp}
\end{align} 
%%%%%%%%%%%%%%%%%

\video{L'algorithme de programmation dynamique exacte}{https://youtu.be/dfz9k3BGrH0}

\subsection{Exemples}

\subsubsection{Système de chauffage optimal}

\video{Exemple de commande optimale pour un système de chauffage}{https://youtu.be/QwXjiAzDENs}


\subsubsection{Navigation optimale}


\newpage
%%%%%%%%%%%%%%%%%%%%%%%%%%%%%%%%%%%%%%%%%%%%%%%%
\subsection{Variations sur un thème de programmation dynamique}

\paragraph{Stochastique}

%%%%%%%%%%%%%%%%
\begin{align}
J^*_k(x_k) = 
\operatornamewithlimits{min}\limits_{u_k}
{\color{red}
\operatornamewithlimits{E}\limits_{w_k}
}
%\mathbb{E}
&\left[
g_k(x_k , u_k , w_k ) + J^*_{k+1}( 
\underbrace{
f_k(x_k , u_k , w_k ) 
}_{x_{k+1}}
)
\right] 
\end{align} 
%%%%%%%%%%%%%%%%%

\paragraph{Robuste}

%%%%%%%%%%%%%%%%
\begin{align}
J^*_k(x_k) = 
\operatornamewithlimits{min}\limits_{u_k}
{\color{red}
\operatornamewithlimits{max}\limits_{w_k}
}
%\mathbb{E}
&\left[
g_k(x_k , u_k , w_k ) + J^*_{k+1}( 
\underbrace{
f_k(x_k , u_k , w_k ) 
}_{x_{k+1}}
)
\right] 
\end{align} 
%%%%%%%%%%%%%%%%%

\paragraph{À horizon de temps infini}

%%%%%%%%%%%%%%%%
\begin{align}
J^*(x) = 
\operatornamewithlimits{min}\limits_{u}
%\mathbb{E}
&\left[
g(x , u ) + {\color{red}\alpha} J^*( 
\underbrace{
f(x , u ) 
}_{x_{k+1}}
)
\right] 
\end{align} 
%%%%%%%%%%%%%%%%%

\paragraph{Sans modèles (apprentissage par renforcement)}

%%%%%%%%%%%%%%%%
\begin{align}
Q^*(x, u ) = g(x , u ) + 
\operatornamewithlimits{min}\limits_{u_{k+1}}
&\left[
Q^*( 
\underbrace{
f(x , u ) 
}_{x_{k+1}}
, u_{k+1}
)
\right] 
\end{align} 
%%%%%%%%%%%%%%%%%

\paragraph{À temps continu}

%%%%%%%%%%%%%%%%
\begin{align}
0 =
\operatornamewithlimits{min}\limits_{u}
%\mathbb{E}
\left[
g(x , u ) + \frac{\partial	J^*(x,t)}{\partial x }
\underbrace{
f(x , u , t) )
}_{\dot{x}}
\right]
\label{eq:hjb}
\end{align} 
%%%%%%%%%%%%%%%%%


\newpage
%%%%%%%%%%%%%%%%%%%%%%%%%%%%%%%%%%%%%%%%%%%%%%%%
\subsection{Approches de programmation dynamique approximée}


%%%%%%%%%%%%%%%%%%%%%%%%%%%%%%%%%%%%%%%%%%%%%%%%%%%%%%%%%%%%%%%%%%%%%%%%%%%%%%%%
\newpage
%%%%%%%%%%%%%%%%%%%%%%%%%%%%%%%%%%%%%%%%%%%%%%%%%%%%%%%%%%%%%%%%%%%%%%%%%%%%%%%%
\section{Problèmes déterministes discrets}
Lorsqu'on travail avec un problème où les états et actions possibles sont discrets, c'est à dire qu'il peuvent prendre un nombre fini de valeurs. Par exemple, pour une automobile la transmission a généralement autour de 6 options discrètes (marche arrière, neutre, 1ère vitesse, 2e vitesse, etc.) tandis que la commande d'accélération est une variable continue qui peut prendre n'importe quel valeur à l'intérieur d'une certaine plage. Pour les systèmes ou tout est discret, il est possible de représenter les états comme des noeuds sur un graphe et les actions possible comme des arcs qui nous mène vers un autre état. Dans ce contexte on peut simplifier la notation.


%%%%%%%%%%%%%%%%
\begin{align}
J^*_k(i) &= 
\operatornamewithlimits{min}\limits_{j \in U(i)}
%\mathbb{E}
\left[
a_{ij}^k + J^*_{k+1}(j)
\right] \\
j^*_k(i) &= 
\operatornamewithlimits{argmin}\limits_{j \in U(i)}
%\mathbb{E}
\left[
a_{ij}^k + J^*_{k+1}(j)
\right] 
\label{eq:exactdpgraph}
\end{align} 
%%%%%%%%%%%%%%%%%

À venir!

\video{Programmation dynamique pour les systèmes discrets et problème du chemin le plus court sur un graphe}{https://youtu.be/QwXjiAzDENs}

\video{Exemple de solution pour le chemin le plus court sur un graphe}{https://youtu.be/1GXUNWVgZOU}


\newpage
%%%%%%%%%%%%%%%%%%%%%%%%%%%%%%%%%%%%%%%%%%%%%%%%%%%%%%%%%%%%%%%%%%%%%%%%%%%%%%%%
\section{Problèmes stochastique}


On cherche donc ici à minimiser l'espérance du coûts-à-venir:
\begin{align}
    J = \large{E} \left\{ \sum_{k=0}^{N-1} g_k(x_k, u_k , w_k) + g_N( x_N ) \right\}
\end{align}
avec l'algorithme de programmation dynamique suivant:
%%%%%%%%%%%%%%%%
\begin{align}
J^*_k(x_k) = 
\operatornamewithlimits{min}\limits_{u_k}
\operatornamewithlimits{E}\limits_{w_k}
%\mathbb{E}
\left[
g_k(x_k , u_k , w_k ) + J^*_{k+1}( 
\underbrace{
f_k(x_k , u_k , w_k ) 
}_{x_{k+1}}
)
\right] \\
u^*_k(x_k) = 
\operatornamewithlimits{argmin}\limits_{u_k}
\operatornamewithlimits{E}\limits_{w_k}
%\mathbb{E}
\left[
g_k(x_k , u_k , w_k) + J^*_{k+1}( 
\underbrace{
f_k(x_k , u_k , w_k) 
}_{x_{k+1}}
)
\right] 
\end{align} 
%%%%%%%%%%%%%%%%%

\video{Commande stochastique}{https://youtu.be/1GXUNWVgZOU}


%%%%%%%%%%%%%%%%%%%%%%%%%%%%%%%%%%%%%%%%%%%%%%%%%%%%%%%%%%%%%%%%%%%%
\subsection{Chaînes de Markov}

\video{Programmation dynamique pour un processus de Markov}{https://youtu.be/xHoLePda478}



À venir!

%%%%%%%%%%%%%%%%%%%%%%%%%%%%%%%%%%%%%%%%%%%%%%%%%%%%%%%%%%%%%%%%%%%%%%%%%%%%%%%%
%%%%%%%%%%%%%%%%%%%%%%%%%%%%%%%%%%%%%%%%%%%%%%%%%%%%%%%%%%%%%%%%%%%%
\section{Robustesse et approche minimax}

À venir!


\newpage
%%%%%%%%%%%%%%%%%%%%%%%%%%%%%%%%%%%%%%%%%%%%%%%%%%%%%%%%%%%%%%%%%%%%%%%%%%%%%%%%
%%%%%%%%%%%%%%%%%%%%%%%%%%%%%%%%%%%%%%%%%%%%%%%%%%%%%%%%%%%%%%%%%%%%
\section{Optimisation sur un horizon de temps infini}

\begin{align}
    J = 
    \lim_{N\rightarrow \infty}
    \e{ \sum_{k=0}^{N} 
    \alpha^k
    g_k
    }
\end{align}

\subsection{Équation de Bellman}

À venir!

\video{Équation de Bellman}{https://youtu.be/WbpSBaChigQ}

\video{Éléments de l'équation de Bellman}{https://youtu.be/18KrNlHHT3E}

\video{Variantes de l'équation de Bellman}{https://youtu.be/98I0SI_jWyY}


\subsection{Méthodes de résolutions}

\video{Aperçu des méthodes pour calculer des lois de commandes optimales}{https://youtu.be/jWP9yiIL7gY}



\subsection{Algorithme d'itération de valeurs (\textit{Value-iteration})}

À venir!

\video{Algorithme d'itération de valeurs}{https://youtu.be/SpWKHB4GupU}

\subsection{Algorithme d'itération de loi de commande (\textit{policy-iteration})}

À venir!

\video{Algorithme d'itération de politiques}{https://youtu.be/5b1o88O8e44}





%%%%%%%%%%%%%%%%%%%%%%%%%%%%%%%%%%%%%%%%%%%%%%%%%%%%%%%%%%%%%%%%%%%%%%%%%%%%%%%%
\newpage
%%%%%%%%%%%%%%%%%%%%%%%%%%%%%%%%%%%%%%%%%%%%%%%%%%%%%%%%%%%%%%%%%%%%
\section{Solution LQR}

Dans le contexte d'un système dynamique linéaire à états/actions continus représenté par:
\begin{align}
    \col{x}_{k+1} = f_k(\col{x}_k , \col{u}_k , \col{w}_k )  = A_k \col{x}_k + B_k \col{u}_k + \col{w}_k
\end{align}
où $\col{x}_k$ et $\col{w}_k$ sont des vecteurs de dimension $n$ et $\col{w}_k$ un vecteur de dimension $m$. Le vecteur $\col{w}_k$ représente des perturbations aléatoires, indépendantes de l'état actuel, de l'action actuelle et de tous les états passés, et avec des distributions centrées autour de zéro: 
\begin{align}
    \e{\col{w}_k} = \col{0} %\quad \quad \e{w_i w_j} = 0 \; \forall \; i \neq j
\end{align}

Si on cherche donc à minimiser l'espérance du coût-à-venir:
\begin{align}
    J = \e{ \sum_{k=0}^{N-1} 
    \left(
    \underbrace{
    \col{x}_k^T Q_k \col{x}_k + \col{u}_k^T R_k \col{u}_k
    }_{ g_k( \col{x}_k, \col{u}_k , \col{w}_k ) }
    \right)
    +
    \underbrace{
    \col{x}_N^T Q_N \col{x}_N
    }_{ g_N( \col{x}_N )  }
    }
\end{align}
où les matrices $Q_k$ et $R_k$ sont symétriques et définies positives:
\begin{align}
    Q_k \geq 0 \quad \quad R_k > 0 
\end{align}

En appliquant l'algorithme de programmation dynamique (pour l'étape $N\rightarrow N-1$ ou une étape générique $k+1 \rightarrow k$), on trouve que:
1) le coût-à-venir d'un état $\col{x}_k$ a la forme quadratique suivante:
\begin{align}
    J_k^*( \col{x}_k ) = \col{x}_k^T S_k \col{x}_k + c
\end{align}
où $S_k$ est une matrice symétrique qui caractérise le coût-à-venir à l'état $\col{x}_k$ et $c$ est une constante qui ne dépent pas de l'état actuel.
2) la loi de commande optimale a la forme linéaire suivante:
\begin{align}
    \col{u}_k^* = c_k^*( \col{x}_k ) = - K_k \col{x}_k
\end{align}
où $K_k$ est la matrice de gain égale à
\begin{align}
    K_k = \left[ R_k + B_k^T S_{k+1} B_k \right]^{-1} B_k S_{k+1} A_k
\end{align}

3) la matrice $S_k$ dans les équations précédentes peut être calculée en partant du coût final à $k=N$ et en remontant dans le temps avec la récursion suivante:
\begin{align}
    S_k = Q_k + A_k^T \left( S_{k+1} - S_{k+1}^T B_k^T  \left[ R_k + B_k^T S_{k+1} B_k \right]^{-1} B_k S_{k+1} \right) A_k
\end{align}


\video{Solution LQR (temps discret)}{https://youtu.be/gg0IYkQwXWs}


%%%%%%%%%%%%%%%%%%%%%%%%%%%%%%%%%%%%%%%%%%%%%%%%%%%%%%%%%%%%%%%%%%%%%%%%%%%%%%%%
\newpage
%%%%%%%%%%%%%%%%%%%%%%%%%%%%%%%%%%%%%%%%%%%%%%%%%%%%%%%%%%%%%%%%%%%%
\section{Apprentissage par renforcement}


À venir!

\video{De la commande optimale à l'apprentissage par renforcement}{https://youtu.be/eyUcnKOpwsE}

\subsection{TD-Learning}

\subsection{Q-Learning}

\subsection{Sarsa}


\subsection{Approximation de fonctions}

\video{Apprentissage par renforcement avec une approximation de fonction}{https://youtu.be/CGbdEDGsJnU}

\video{Méthodes d'approximation de fonctions}{https://youtu.be/p8BizsV8apQ}



%%%%%%%%%%%%%%%%%%%%%%%%%%%%%%%%%%%%%%%%%%%%%%%%%%%%%%%%%%%%%%%%%%%%%%%%%%%%%%%%
\newpage
%%%%%%%%%%%%%%%%%%%%%%%%%%%%%%%%%%%%%%%%%%%%%%%%%%%%%%%%%%%%%%%%%%%%
\section{Formulation en temps continu}
\label{sec:dp_cont}

À venir!

\subsection{Hamilton–Jacobi–Bellman equation}

%%%%%%%%%%%%%%%%
\begin{align}
0 =
\operatornamewithlimits{min}\limits_{\col{u}}
%\mathbb{E}
\left[
g(\col{x} , \col{u} ) + \frac{\partial	J^*(\col{x},t)}{\partial \col{x} }
\underbrace{
f(\col{x} , \col{u} , t) )
}_{\dot{\col{x}}}
\right]
\label{eq:hjb}
\end{align} 
%%%%%%%%%%%%%%%%%
